We have found what are the difference multisets if the parameter values are small. Here we present a list of our findings. Some trivial cases that formally satisfy the constraints (e.g. some produce every element 0 times) and quasigroup cases are also included as those have helped spotting patterns.

\begin{tabular}{llr}
\toprule
    Parameters & Difference multisets \\
\midrule
    $k = 0$ & Empty multiset. \\
    $k = 1$ & Take single element, works for $v \geq 1$. \\
    $v = 0$ & Empty multiset. \\
    $v = 1$ & Take the identity $k$ times for any $k$. \\
    $v = 2$ & Covered in Section \ref{sec:z2i} as a case of $\mathbb Z_2^i$. \\
    $v = 3$ & See section \ref{sec:v3}. \\
    $G=\mathbb Z_2^i$ & See section \ref{sec:z2i}, possibly incomplete. \\
    $G=\mathbb Z_3$ & See section \ref{sec:z3}. \\
\bottomrule
\end{tabular}


The case of difference multisets over $\mathbb Z_3$ (theorem \ref{v3:theorem:loeschian}) shows that not only the very trivial cases can be solved explicitly. Although it is not straightforward to generalize our methods for arbitrary $\mathbb Z_i$, solving the problem for odd prime values of $i$ seems in the realm of possibility.

Theorem \ref{general:theorem:limits} greatly narrows the space of options that has to be considered in computer searches thus allowing to inspect a wide range of difference multisets and draw conclusions through observations.

The mathematical apparatus we used is also applicable for many other cases. The results presented in this paper are the ones that are in some sense complete or general. Other than these cases the system \eqref{apparatus:eq:dsystem} (or alternative forms) can be utilised to find some difference multisets or sets of their digressions in many small quasigroups. 
