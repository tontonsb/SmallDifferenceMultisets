\begin{theorem}
	\label{regular:theorem:regular}
	If $\sqrt k$ is integer and congruent to $0$ or $\pm 1 \pmod v$ then a $(G,k)$-difference multiset exists with the following digressions.
		\begin{itemize}
			\item If $\sqrt k \equiv 1 \pmod v$ then $d_\mu = v-1$ for any single element $\mu$ and $d_{\nu \neq \mu} = -1$ for the other elements.
			\item If $\sqrt k \equiv -1 \pmod v$ then $d_\mu =1-v$ for any single element $\mu$ and $d_{\nu \neq \mu} = 1$ for the other elements.
			\item Both of the above constructions if $\sqrt k \equiv 0 \pmod v$.
		\end{itemize}
\end{theorem}

\begin{proof}
	The conditions on the values of $k$ and $d_\mu$ guarantee that the multiplicities $n_\mu=\frac{k+d_\mu \sqrt k}v$ are integers. It is left to demonstrate that they form a difference multiset.
	
	Considering equation \eqref{apparatus:eq:dsystem} for non-identity elements we can notice that the digression $\pm(v-1)$ (as any other digression) is involved in two of the products---once as $d_\mu$ and once as $d_{\gamma+\mu}$.

	Other $v-2$ products are $(\pm1)\cdot(\pm1)$, thus the condition is satisfied:
	
	\begin{equation}
		\sum d_\mu d_{\gamma+\mu} = 2(\pm(v-1)\cdot(\mp 1)) + (v-2)(\mp1)^2 = -v.
	\end{equation}

	The condition for the identity $\gamma=0$ is also satisfied:
	
	\begin{equation}
		\sum d_\mu^2  = \left( \pm (v-1) \right)^2 + (v-1) \left( \mp 1 \right)^2 = v^2 - v.
	\end{equation}

	It is straightforward to check that equation \eqref{apparatus:eq:di} is satisfied as well.
\end{proof}
