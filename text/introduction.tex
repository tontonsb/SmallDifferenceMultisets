\subsection{Difference multisets}

    Difference multiset is a combinatorial design similar to difference set. %But a multiset. 

    The classical difference set $D$ in a finite group $G$ is such a subset of $G$ that produces every non-zero $\gamma \in G$ the same number of times when taking the differences between elements of $D$. A simple example is $\set{0,1} \subset \mathbb Z_3$ as $1-0=1$ and $0-1=2$ thus producing both of the non-zero elements of $\mathbb Z_3$. 
%Curiosly, the same pair is also a difference set in $\mathbb Z_2$, but that's boring as $G$ is always a difference set of $G$. 
A less trivial and more famous example is the set $\set{1,2,4} \subset \mathbb Z_7$.

    If we take a multiset instead, we can produce the whole $G$, including the identity element. For example, considering the differences between elements of $\set{0,0,1} \subset \mathbb Z_3$ we obtain $\set{0,0,1,1,2,2}$. This is what is called a difference multiset. Note that we take differences between pairs of elements not an element and itself (i.e. there was $0-0=0$ as first zero subtracted from the second zero and vice versa but not the first zero from itself and no $1-1$).

    While difference sets have been studied at least since 1939 \cite{bose1939construction}, difference multisets were first investigated on their own in 1999 by Buratti \cite{buratti1999old} who noticed that such designs (and the related strong difference families) are indirectly used by other authors in constructions of various combinatorial designs. The paper defined the concept of a difference multiset and presented its basic properties and some constructions. The topic was developed further by other authors \cite{arasu2005cyclic, arasu2005regular} (they used the term ``regular difference covers'' instead of ``difference multisets'') who introduced new constructions and several nonexistence theorems.

    The results in the foundational articles are mostly analogous to those of difference sets, almost all of the constructions are based on some difference set construction. As a result the number of difference multisets constructed in a given finite group is proportional to that of difference sets which is unlikely to reflect the real situation as there are infinitely more multisets over a given finite $G$ than there are subsets. Some constructions producing difference multisets of arbitrary size over fixed $G$ were presented in \cite{momihara2009strong} and we strive to expand in this direction---constructing arbitrarily large multisets in fixed, mainly small algebraic structures.

\subsection{Synopsis}

    We study the difference multisets using a system of quadratic and linear equations on the multiplicities of their elements. We show that these multiplicities of any difference multiset over a loop are in a sense close to their average. This leads to the next idea of studying their digressions from the average which allows describing difference multisets with a simpler equation system.

    Using these tools we find a construction that allows to produce an infinite number of difference multisets over any group. Focusing on groups $\mathbb Z_2^i$, we obtain a more general construction that includes the previous one as a special case, and also provides all multisets for $\mathbb Z_2^i$ for at least $i \leq 3$.

    We also solve the problem of difference multisets of all quasigroups of cardinality 3. Interestingly, the possible sizes of difference multisets over $\mathbb Z_3$ turn out to be Löschian numbers.
    
