Let us consider \eqref{apparatus:eq:dsystem} in matrix form 
\begin{equation}
\label{general:eq:matrix_eq}
	D d = \bf v
\end{equation}
where $d = (d_0, d_1, \ldots, d_{v-1})^T$, $\bf{v} = (v^2-v, -v, -v, \ldots)^T$ and $D_{\mu\nu} = d_{\mu+\nu}$, i.e.\ it is the Cayley table of the group in question.

For cyclic groups the matrix $D$ takes the form of an \emph{anticirculant} or \emph{left circulant} matrix:

\begin{equation}
	\label{general:eq:anticirculant_matrix}
	D =
	\begin{pmatrix}
		d_0 & d_1 & d_2 & \cdots & d_{v-1} \\ 
		d_1 & d_2 & d_3 & \cdots & d_0 \\
		d_2 & d_3 & d_4 & \cdots & d_1 \\
		\vdots & \vdots & \vdots & \ddots & \vdots \\
		d_{v-2} & d_{v-1} & d_0 & \cdots & d_{v-3} \\
		d_{v-1} & d_0 & d_1 & \cdots & d_{v-2} \\
	\end{pmatrix}.
\end{equation}

We will use the discrete Fourier transform:
\begin{equation}
	\mathcal F (y)_m = \sum_{j=0}^{v-1} y_j \omega^{-jm}
\end{equation}
where $\omega = \exp(\frac{2\pi \imath}v)$. Let us consider an equation $Ax=b$ with anticirculant matrix $A$. Applying the discrete Fourier transform to $b$ we obtain

\begin{equation}
	\label{general:eq:fourier_image}
	\mathcal F (b)_m = \mathcal F (a)_m \mathcal F ^*(x^*)_m,
\end{equation}
where $a$ is the first row of $A$.

This is particulary useful for equation \eqref{general:eq:matrix_eq} as the vector $d$ is also the first row of matrix $D$. The image of $\bf v=Dd$ is

\begin{equation}
	\mathcal{F} ({\bf v})_m = \mathcal{F}(d)_m \mathcal{F^*}(d^*)_m.
\end{equation}

As we are only interested in real $d$, we can simplify it further:

\begin{equation}
	\label{general:eq:dfourier}
	\mathcal{F} ({\bf v})_m = \mathcal{F}(d)_m \mathcal{F^*}(d)_m = |\mathcal{F}(d)_m|^2.
\end{equation}

Since ${\bf v} = (v^2-v, -v, -v, \ldots)^T$ we can find that $\mathcal{F}({\bf v}) = (0,v^2,v^2,\ldots)$ and \eqref{general:eq:dfourier} becomes
\begin{equation}
	\label{gen:eq:dfourierfinal}
	\left| \sum_{\mu=0}^{v-1} d_\mu \omega^{-\mu m} \right| = v (1-\delta_{m0}).
\end{equation}

\begin{theorem}
	\label{general:theorem:split_cycles}
	The digressions of a $(\mathbb Z_v, k)$-difference multiset are related via equation
	\begin{equation}
		\label{general:eq:split_cycles}
		\left| 
			\sum_{\kappa=0}^{v/m-1} \omega^{-m\kappa} \cdot 
			\sum_{\nu=0}^{m-1}  d_{\kappa+\nu v/m} 
		\right| = v
	\end{equation}
	where $m$ is any divisor of $v$ and $\omega = \exp(\frac{2\pi \imath}v)$.
\end{theorem}

\begin{proof}
	The theorem statement is obtained from \eqref{gen:eq:dfourierfinal}.
	
	Select a divisor $m$ of $v$. Then $\delta_{m0}=0$ on the right hand side. The left hand side is transformed and grouped into smaller cycles using the following relation:
	\begin{equation}
		\omega^{-m (\kappa+\nu v/m)}
		= \omega^{-m \kappa} \omega^{-\nu v}
		= \omega^{-m \kappa}.
	\end{equation}
\end{proof}

\begin{proposition}
	\label{general:theorem:even_cyclic}
	In difference multistes over cyclic groups of even cardinality $\sum_{\mu=0}^{v/2-1} d_{2\mu} = \pm \frac v2$ and $\sum_{\mu=0}^{v/2-1} d_{2\mu+1} = \mp \frac v2$.
\end{proposition}
\begin{proof}
	Take \eqref{general:eq:split_cycles} for $m=v/2$
	\begin{equation}
		\left| 
			\sum_{\mu=0}^{v/2-1} d_{2\mu} 
			- \sum_{\mu=0}^{v/2-1} d_{2\mu+1} 
		\right| = v.
	\end{equation}
	
	I.e.\ the total of even $d_\mu$'s and the total of odd $d_\mu$'s differ by $v$. Take into account that the grand total is $\sum d_\mu = 0$ and the proposition follows.
\end{proof}

\begin{remark}
	Similar relation also holds true for some other structures that are 
	not cyclic groups. For example in $\mathbb Z_2 \times \mathbb Z_2$
	with elements $\set{\mu, \nu, \zeta, \eta}$ in any order we have 
	$(d_\mu+d_\nu)-(d_\zeta+d_\eta)=\pm 4$.
\end{remark}
