Consider $\mathbb Z_2^i$ as an affine space over $\mathbb Z_2$. For a chosen affine frame $F$, each element $\mu\in\mathbb Z_2^i$ can be represented by its affine coordinates: $\mu=(\mu_1, \mu_2, \ldots, \mu_i)$. For $r\in\{1,2,\ldots,i\}$ we define subspaces $H_r^F$ and $\widetilde H_r^F$ as follows:

\begin{equation}
    \mu \in H_r^F \iff 
        \begin{cases}
            \mu_j = 1 & \text{for } j < r \\
            \mu_r = 0
        \end{cases}
\end{equation}

\begin{equation}
    \mu \in \widetilde H_r^F \iff 
        \mu_j = 1 \text{ for } j \leq r
\end{equation}
and denote $\xi_r = \sum\limits_{j=0}^r (-2)^j=(1-(-2)^{r+1})/3$.

\begin{theorem}
    \label{z2i:theorem:construction}
    For an arbitrary affine frame $F$ of $\mathbb Z_2^i$ and integer $m \colon 0 < m < i$, the following construction produces a $(\mathbb Z_2^i, k)$-difference multiset iff $k$ is square and $v | \sqrt k$:
    \begin{enumerate}
        \item For each $\mu \in H_r^F$ where $r \leq m$, set $d_\mu = \xi_r$.
        \item Select element $\nu \in \widetilde H_m^F$.
        \item Set $d_\nu = (-1)^m v + \xi_{m+1}$.
        \item Set $d_\mu = \xi_{m+1}$ for all $\mu \in \widetilde H_m^F$ such that $\mu \neq \nu$.
    \end{enumerate}
\end{theorem}

A difference multiset is also obtained if the opposite sign is used on every $d_\mu$.

\begin{proof}
	For a selected $m \colon 0 < m < i$ this construction provides us with $2^{i-r}$ digressions of value $d_\mu=\xi_r$ for each $1 \leq r\leq m$ (none of these if $m=0$), $2^{i-m}-1$ digressions equal to $\xi_{m+1}$ and one digression equal to $(-1)^m 2^i+\xi_{m+1}$.
    
    Equations $\sum d_\mu = 0$ and $\sum d_\mu^2 = v(v-1)$ can be shown to be satisfied inserting the values listed above.
    
    Equations \eqref{apparatus:eq:dsystem} are left to check for non-identity $\gamma$.
    
    Having $\gamma = (\gamma_1, \gamma_2, \ldots, \gamma_i)$ with $\gamma_g = 1$ for some index $g$ and $\gamma_r = 0$ for all $r < g$ we can observe that:
    \begin{itemize}
        \item $\mu \in H_r^F \iff \gamma + \mu \in H_r^F$ if $r < g \lor g > m$;
        \item $\mu \in H_r^F \iff \gamma + \mu \in H_g^F$ if $r > g \land g \leq m$;
        \item $\mu \in H_g^F \iff \gamma + \mu \in H_r^F$ where $r>g$ if $g \leq m$.
    \end{itemize}
    
    The $\widetilde H_m^F$ also counts as one of the $H_r$ subsets whenever $r<g$ is not required to be true.
    
    We shall consider $g\leq m$ first. The value of \eqref{apparatus:eq:dsystem} is:
    
    \begin{equation}
        \begin{split}
            \sum d_\mu d_{\gamma+\mu}
              = & \sum\limits_{r<g} \sum\limits_{\mu \in H_r} d_\mu d_{\gamma + \mu}
                + \sum\limits_{r>g} \sum\limits_{\mu \in H_r} d_\mu d_{\gamma + \mu}
                + \sum\limits_{\mu \in H_g} d_\mu d_{\gamma + \mu} \\
              = & \sum\limits_{r<g} \sum\limits_{\mu \in H_r} \xi_r^2
                + 2\sum\limits_{r>g} \sum\limits_{\mu \in H_r} \xi_g d_\mu \\
              = & \sum\limits_{r<g} \sum\limits_{\mu \in H_r} \xi_r^2
                - 2 \xi_g \sum\limits_{r \leq g} \sum\limits_{\mu \in H_r} \xi_r \\
              = & \sum\limits_{r<g} 2^{i-r} \xi_r^2
                - 2 \xi_g \sum\limits_{r \leq g}  2^{i-r} \xi_r \\
              = & - v.
        \end{split}
    \end{equation}
    
    Considering $g > m$ we get the following:
    
    \begin{equation}
        \begin{split}
            \sum d_\mu d_{\gamma+\mu}
              = & \sum\limits_{r \leq m} \sum\limits_{\mu \in H_r} d_\mu
                + \sum\limits_{\mu \in F} d_\mu d_{\gamma + \mu} \\
              = & \sum\limits_{r\leq m} 2^{i-r} \xi_r^2
                + (2^{i-m}-2) \xi_{m+1}^2 + 2\xi_{m+1}((-1)^m 2^i + \xi_{m+1}) \\
              = & -v.
        \end{split}
    \end{equation}
    
    Lastly, by inserting $d_\mu = \xi_r$ into $n_\mu=\frac{k+d_\mu \sqrt k}v$ we can observe that integer $n_\mu$ is produced iff $v | \sqrt k$.
\end{proof}

\begin{remark}
    We could also allow to select $m=0$ in Theorem \ref{z2i:theorem:construction}. In that case the constructed difference multiset would be the same one encountered in Section \ref{sec:uni} and it would also produce  integer multiplicities for $k \equiv 1 \mod v$ or $k \equiv -1 \mod v$.
\end{remark}

\begin{theorem}
    There are no other difference multisets than those presented in Theorem \ref{z2i:theorem:construction} and Theorem \ref{regular:theorem:regular} for $i < 4$.
\end{theorem}

\begin{proof}
    This result was found directly solving equations \eqref{apparatus:eq:di} and \eqref{apparatus:eq:dsystem}.
    
    For $\mathbb Z_2$ it has already been shown before\cite{buratti1999old}.
    
    For $\mathbb Z_2 \times \mathbb Z_2$ it can also be done by hand.
    
    For the case of $\mathbb Z_2^3$ we accomplished it with the help of a computer algebra system. The single-variable polynomial of Gröbner basis in the case of $\mathbb Z_2^3$ is
    \begin{equation}
        d_\mu^8 - 84 d_\mu^6 + 1974 d_\mu^4 - 12916 d_\mu^2 + 11025
    \end{equation}
    for which the roots are $\pm\xi_1, \pm\xi_2, \pm\xi_3, \pm(8-\xi_1)$.
\end{proof}


Sekojošais jāpārdomā, ja vispār lietderīgi iekļaut.

The construction provided in this section produces
\begin{equation}
    2 \sum\limits_{j=2}^i 2^j \prod\limits_{l=j+1}^i (2^{l+1}-2)
\end{equation}
solutions for $d_\mu$ over $\mathbb Z_2^i$. The counting argument is that we can select $H_1$ in $2^{i+1}-2$ ways, $H_2$ in $2^i-2$ etc. until you stop and choose one of the remaining $2^j$ elements. And twice everything as you can flip the signs. Beware that taking $m=i-1$ will reproduce the difference multisets that can be obtained by taking $m=i-2$.

\begin{example}
    Let us take $i=4$ and $m=2$. We get half the digressions equal to $\xi_1=-1$. Set another quarter---four digressions to $\xi_2=3$. Out of the last four we assign one $v+\xi_3=11$ and $\xi_3=-5$ for the other three.
    
    Let us now take $i=4$ and $m=3$ instead. The first twelve digressions are assigned in the same way. Half of the remaining four are set to $\xi_3=-5$. Then we select one out of the remaining subset to set it to $-v+\xi_4=-5$ and all the remaining (i.e.\ one) to $=\xi_4=11$. Thus we end up with the same set of digressions as in the previous case.
\end{example}
