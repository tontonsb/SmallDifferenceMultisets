Consider $\mathbb Z_2^m$ as an affine space over $\mathbb Z_2$. For a chosen affine frame $F$, each element $\mu\in\mathbb Z_2^m$ can be represented by its affine coordinates: $\mu=(\mu_1, \mu_2, \ldots, \mu_m)$. For $i\in\{1,2,\ldots,m\}$ we define subspaces $H_i^F$ and $\widetilde H_i^F$ as follows:

\begin{equation}
    \mu \in H_i^F \iff 
        \begin{cases}
            \mu_j = 1 & \text{for } j < i \\
            \mu_i = 0
        \end{cases}
\end{equation}

\begin{equation}
    \mu \in \widetilde H_i^F \iff 
        \mu_j = 1 \text{ for } j \leq i
\end{equation}
and denote $\xi_i = \sum\limits_{j=0}^i (-2)^j=(1-(-2)^{i+1})/3$.

\begin{theorem}
    \label{z2i:theorem:construction}
    For an arbitrary affine frame $F$ of $\mathbb Z_2^m$ and integer $r \colon 0 < r < m$, the following construction produces a $(\mathbb Z_2^m, k)$-difference multiset iff $k$ is square and $v | \sqrt k$:
    \begin{enumerate}
        \item For each $\mu \in H_i^F$ where $i \leq r$, set $d_\mu = \xi_i$.
        \item Select element $\nu \in \widetilde H_r^F$.
        \item Set $d_\nu = (-1)^r v + \xi_{r+1}$.
        \item Set $d_\mu = \xi_{r+1}$ for all $\mu \in \widetilde H_r^F\setminus\{\nu\}$.
    \end{enumerate}
    A difference multiset is also obtained if the opposite sign is used on every $d_\mu$.
\end{theorem}

\begin{proof}
	For a selected $r \colon 0 < r < m$ this construction provides us with $2^{m-i}$ digressions of value $d_\mu=\xi_i$ for each $i$ ($1 \leq i\leq r$), $2^{m-r}-1$ digressions equal to $\xi_{r+1}$ and one digression equal to $(-1)^r 2^m+\xi_{r+1}$.
    
    It is easy to check that the equations $\sum d_\mu = 0$ and $\sum d_\mu^2 = v(v-1)$ are satisfied with these values.
    
    It remains to check the equations \eqref{apparatus:eq:dsystem} for non-identity $\gamma$.
    
    For $\gamma = (\gamma_1, \gamma_2, \ldots, \gamma_m)$ let $s$ be the smallest index for which $\gamma_s=1$. Then we can observe the following behaviour of $\gamma + \mu$:
    \begin{itemize}
        \item If $s > r$ then $\mu \in H_i^F \iff  \gamma + \mu \in H_i^F$ and $\mu \in \widetilde H_r^F \iff  \gamma + \mu \in \widetilde H_r^F$.
        \item If $s \leq r \land i < s$ for then $\mu \in H_i^F \iff  \gamma + \mu \in H_i^F$.
        \item If $s \leq r \land i = s$ then $\mu \in H_i^F \iff  \gamma + \mu \in \widetilde H_i^F$.
        \item If $s \leq r \land i > s \land i \leq r$ then $\mu \in H_i^F \iff  \gamma + \mu \in H_s^F$.
        \item If $s \leq r$ then $\mu \in \widetilde H_r^F \iff \gamma + \mu \in H_s^F$
    \end{itemize}
    
    We shall consider the case $s\leq r$ first. The value of \eqref{apparatus:eq:dsystem} is evaluated as follows
    
    \begin{equation}
        \begin{split}
            \sum d_\mu d_{\gamma+\mu}
              = & \sum\limits_{i<s} \sum\limits_{\mu \in H_i^F} d_\mu d_{\gamma + \mu}
                + \sum\limits_{i>s} \sum\limits_{\mu \in H_i^F} d_\mu d_{\gamma + \mu}
                + \sum\limits_{\mu \in H_s^F} d_\mu d_{\gamma + \mu}. \\
        \end{split}
    \end{equation}
    
    In the second and third sum one of $\mu$ and $\gamma + \mu$ belongs to $H_s^F$ and the other belongs to $H_i^F$,
thus either $d_\mu$, or $d_{\gamma+\mu}$ is equal to $\xi_s$:
    
    \begin{equation}
        \begin{split}
            \sum d_\mu d_{\gamma+\mu}
              = & \sum\limits_{i<s} \sum\limits_{\mu \in H_i^F} \xi_i^2
                + 2\sum\limits_{i>s} \sum\limits_{\mu \in H_i^F} \xi_s d_\mu. \\
        \end{split}
    \end{equation}
    
    Since $\sum d_{\mu} = 0$, we can replace $\sum_{i>s} \sum_{\mu \in H_i^F} d_\mu$ with $-\sum_{i\leq s} \sum_{\mu \in H_i^F} d_\mu$ and substitute $d_\mu=\xi_i$:
    
    \begin{equation}
        \begin{split}
            \sum d_\mu d_{\gamma+\mu}
              = & \sum\limits_{i<s} \sum\limits_{\mu \in H_i^F} \xi_i^2
                - 2 \xi_s \sum\limits_{i \leq s} \sum\limits_{\mu \in H_i^F} \xi_i \\
              = & \sum\limits_{i<s} 2^{m-i} \xi_i^2
                - 2 \xi_s \sum\limits_{i \leq s}  2^{m-i} \xi_i \\
              = & - v.
        \end{split}
    \end{equation}
    
    Considering the other case with $s > r$ we get the following:
    
    \begin{equation}
        \begin{split}
            \sum d_\mu d_{\gamma+\mu}
              = & \sum\limits_{i \leq r} \sum\limits_{\mu \in H_i^F} d_\mu
                + \sum\limits_{\mu \in \widetilde H_r^F} d_\mu d_{\gamma + \mu} \\
              = & \sum\limits_{i\leq r} 2^{m-i} \xi_i^2
                + (2^{m-r}-2) \xi_{r+1}^2 + 2\xi_{r+1}((-1)^r 2^m + \xi_{r+1}) \\
              = & -v.
        \end{split}
    \end{equation}
    
    Lastly, by inserting $d_\mu = \xi_i$ into $n_\mu=\frac{k+d_\mu \sqrt k}v$ we can observe that $n_\mu$ is integer iff $v | \sqrt k$.
\end{proof}

\begin{remark}
    We could also allow the value $r=0$ in Theorem \ref{z2i:theorem:construction}. In that case the obtained difference multiset would be the one described in 
Theorem \ref{regular:theorem:regular} and it would also produce  integer multiplicities for $k \equiv 1 \pmod v$ or $k \equiv -1 \pmod v$.
\end{remark}

\begin{theorem}
    There are no other difference multisets than those presented in Theorem \ref{z2i:theorem:construction} and Theorem \ref{regular:theorem:regular} for $m < 4$.
\end{theorem}

\begin{proof}
    This result was found by solving equations \eqref{apparatus:eq:di} and \eqref{apparatus:eq:dsystem}.
    
    For $\mathbb Z_2$ it has already been shown before by Buratti \cite{buratti1999old}.
    
    For $\mathbb Z_2 \times \mathbb Z_2$ our constructions produce a single digression equal to $\pm 3$ and the other three equal to $1$. Equations \eqref{apparatus:eq:di} and \eqref{apparatus:eq:dsystem} take the following form:
    
    \begin{equation}
        \label{z2i:eq:z2z2}
        \begin{cases}
            d_{00} + d_{01} + d_{10} + d_{11} = 0 \\
            d_{00}^2 + d_{01}^2 + d_{10}^2 + d_{11}^2 = 12 \\
            2 d_{00}d_{01} + 2 d_{10}d_{11} = -4 \\
            2 d_{00}d_{10} + 2 d_{01}d_{11} = -4 \\
            2 d_{00}d_{11} + 2 d_{01}d_{10} = -4 \\
        \end{cases}.
    \end{equation}
    
    Clearly not all $d_\mu$ are $0$ and at least one digression is positive, at least one is negative. From the latter three equations we can observe that it is impossible to have two positive and two negative digressions so there is exactly one positive or exactly one negative digression. WLOG suppose $d_{00}$ is the only positive digression.
    
    It is easy to evaluate that $(d_{00}+d_{01})^2+(d_{01}+d_{11})^2=8$ using the above equations. Using $d_{01}+d_{11} = -(d_{00}+d_{01})$ from the first equation we obtain $2(d_{00}+d_{01})^2=8$ and $d_{00}+d_{01} = \pm 2$. The same can be shown about any pair of digressions.
    
    Three times the first equation of system \eqref{z2i:eq:z2z2} can be rewritten as follows:
    \begin{equation*}
        (d_{00}+d_{01}) + (d_{00}+d_{10}) + (d_{00}+d_{11}) + (d_{01}+d_{10}) + (d_{01}+d_{11}) + (d_{10}+d_{11}) = 0.
    \end{equation*}
    
    We know that the summands are equal to $\pm 2$ so three of them must be equal to $+2$ and three to $-2$. The "$+2$" ones are the first three as they need to contain the positive digression $d_{00}$. The other digressions are then obviously equal and of value $-1$ as their pairs sum to $-2$. And $d_{00}$ is then $3$.
    
    If we had supposed there is a single negative digression, we would have obtained it to be $-3$ and the others would be $1$.
    
    For $\mathbb Z_2^3$ equations \eqref{apparatus:eq:di} and \eqref{apparatus:eq:dsystem} take the following form:
    
    \begin{equation}
        \begin{cases}
            d_{000} + d_{001} + d_{010} + d_{011} + d_{100} + d_{101} + d_{110} + d_{111} = 0 \\
            d_{000}^2 + d_{001}^2 + d_{010}^2 + d_{011}^2 + d_{100}^2 + d_{101}^2 + d_{110}^2 + d_{111}^2 = 56 \\
            2 d_{000}d_{001} + 2 d_{010}d_{011} + 2 d_{100}d_{101} + 2 d_{110}d_{111} = -8 \\
            2 d_{000}d_{010} + 2 d_{001}d_{011} + 2 d_{100}d_{110} + 2 d_{101}d_{111} = -8 \\
            2 d_{000}d_{011} + 2 d_{001}d_{010} + 2 d_{100}d_{111} + 2 d_{101}d_{110} = -8 \\
            2 d_{000}d_{101} + 2 d_{001}d_{100} + 2 d_{010}d_{111} + 2 d_{011}d_{110} = -8 \\
            2 d_{000}d_{110} + 2 d_{001}d_{111} + 2 d_{010}d_{100} + 2 d_{011}d_{101} = -8 \\
            2 d_{000}d_{111} + 2 d_{001}d_{110} + 2 d_{010}d_{101} + 2 d_{011}d_{100} = -8 \\
        \end{cases}
    \end{equation}
    
    We solved this case with the help of a computer algebra system. The only solutions to this system are the ones produced in Theorem \ref{z2i:theorem:construction} and Theorem \ref{regular:theorem:regular}.
\end{proof}


% The construction provided in this section produces
% \begin{equation}
%     2 \sum\limits_{j=2}^m 2^j \prod\limits_{l=j+1}^m (2^{l+1}-2)
% \end{equation}
% solutions for $d_\mu$ over $\mathbb Z_2^m$. The counting argument is that we can select $H_1$ in $2^{m+1}-2$ ways, $H_2$ in $2^m-2$ etc. until you stop and choose one of the remaining $2^j$ elements. And twice everything as you can flip the signs. Beware that taking $i=m-1$ will reproduce the difference multisets that can be obtained by taking $i=m-2$.
% 
% \begin{example}
%     Let us take $m=4$ and $i=2$. We get half the digressions equal to $\xi_1=-1$. Set another quarter---four digressions to $\xi_2=3$. Out of the last four we assign one $v+\xi_3=11$ and $\xi_3=-5$ for the other three.
%     
%     Let us now take $m=4$ and $i=3$ instead. The first twelve digressions are assigned in the same way. Half of the remaining four are set to $\xi_3=-5$. Then we select one out of the remaining subset to set it to $-v+\xi_4=-5$ and all the remaining (i.e.\ one) to $=\xi_4=11$. Thus we end up with the same set of digressions as in the previous case.
% \end{example}
