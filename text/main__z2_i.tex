Consider $\mathbb Z_2^m$ as an affine space over $\mathbb Z_2$. For a chosen affine frame $F$, each element $\mu\in\mathbb Z_2^m$ can be represented by its affine coordinates: $\mu=(\mu_1, \mu_2, \ldots, \mu_m)$. For $r\in\{1,2,\ldots,m\}$ we define subspaces $H_r^F$ and $\widetilde H_r^F$ as follows:

\begin{equation}
    \mu \in H_r^F \iff 
        \begin{cases}
            \mu_j = 1 & \text{for } j < r \\
            \mu_r = 0
        \end{cases}
\end{equation}

\begin{equation}
    \mu \in \widetilde H_r^F \iff 
        \mu_j = 1 \text{ for } j \leq r
\end{equation}
and denote $\xi_r = \sum\limits_{j=0}^r (-2)^j=(1-(-2)^{r+1})/3$.

\begin{theorem}
    \label{z2i:theorem:construction}
    For an arbitrary affine frame $F$ of $\mathbb Z_2^m$ and integer $i \colon 0 < i < m$, the following construction produces a $(\mathbb Z_2^m, k)$-difference multiset iff $k$ is square and $v | \sqrt k$:
    \begin{enumerate}
        \item For each $\mu \in H_r^F$ where $r \leq i$, set $d_\mu = \xi_r$.
        \item Select element $\nu \in \widetilde H_i^F$.
        \item Set $d_\nu = (-1)^i v + \xi_{i+1}$.
        \item Set $d_\mu = \xi_{i+1}$ for all $\mu \in \widetilde H_i^F\setminus\{\nu\}$.
    \end{enumerate}
    A difference multiset is also obtained if the opposite sign is used on every $d_\mu$.
\end{theorem}

\begin{proof}
	For a selected $i \colon 0 < i < m$ this construction provides us with $2^{m-r}$ digressions of value $d_\mu=\xi_r$ for each $r$ ($1 \leq r\leq i$), $2^{m-i}-1$ digressions equal to $\xi_{i+1}$ and one digression equal to $(-1)^i 2^m+\xi_{i+1}$.
    
    It is easy to check that the equations $\sum d_\mu = 0$ and $\sum d_\mu^2 = v(v-1)$ are satisfied with these values.
    
    It remains to check the equations \eqref{apparatus:eq:dsystem} for non-identity $\gamma$.
    
    For $\gamma = (\gamma_1, \gamma_2, \ldots, \gamma_m)$ let $s$ be the smallest index for which $\gamma_s=1$. Then we can observe the following behaviour of $\gamma + \mu$:
    \begin{itemize}
        \item If $s > i$ then $\mu \in H_r^F \iff  \gamma + \mu \in H_r^F$ and $\mu \in \widetilde H_i^F \iff  \gamma + \mu \in \widetilde H_i^F$.
        \item If $s \leq i \land r < s$ for then $\mu \in H_r^F \iff  \gamma + \mu \in H_r^F$.
        \item If $s \leq i \land r = s$ then $\mu \in H_r^F \iff  \gamma + \mu \in \widetilde H_r^F$.
        \item If $s \leq i \land r > s$ then $\mu \in H_r^F \iff  \gamma + \mu \in H_s^F$.
        \item If $s \leq i$ then $\mu \in \widetilde H_i^F \iff \gamma + \mu \in H_s^F$
    \end{itemize}
    
    We shall consider the case $s\leq i$ first. The value of \eqref{apparatus:eq:dsystem} is evaluated as follows
    
    \begin{equation}
        \begin{split}
            \sum d_\mu d_{\gamma+\mu}
              = & \sum\limits_{r<s} \sum\limits_{\mu \in H_r^F} d_\mu d_{\gamma + \mu}
                + \sum\limits_{r>s} \sum\limits_{\mu \in H_r^F} d_\mu d_{\gamma + \mu}
                + \sum\limits_{\mu \in H_s^F} d_\mu d_{\gamma + \mu}. \\
        \end{split}
    \end{equation}
    
    In the second and third sum one of $\mu$ and $\gamma + \mu$ belongs to $H_s^F$ and the other belongs to $H_r^F$,
thus either $d_\mu$, or $d_{\gamma+\mu}$ is equal to $\xi_s$:
    
    \begin{equation}
        \begin{split}
            \sum d_\mu d_{\gamma+\mu}
              = & \sum\limits_{r<s} \sum\limits_{\mu \in H_r^F} \xi_r^2
                + 2\sum\limits_{r>s} \sum\limits_{\mu \in H_r^F} \xi_s d_\mu. \\
        \end{split}
    \end{equation}
    
    Since $\sum d_{\mu} = 0$, we can replace $\sum_{r>s} \sum_{\mu \in H_r^F} d_\mu$ with $-\sum_{r\leq s} \sum_{\mu \in H_r^F} d_\mu$ and substitute $d_\mu=\xi_r$:
    
    \begin{equation}
        \begin{split}
            \sum d_\mu d_{\gamma+\mu}
              = & \sum\limits_{r<s} \sum\limits_{\mu \in H_r^F} \xi_r^2
                - 2 \xi_s \sum\limits_{r \leq s} \sum\limits_{\mu \in H_r^F} \xi_r \\
              = & \sum\limits_{r<s} 2^{m-r} \xi_r^2
                - 2 \xi_s \sum\limits_{r \leq s}  2^{m-r} \xi_r \\
              = & - v.
        \end{split}
    \end{equation}
    
    Considering the other case with $s > i$ we get the following:
    
    \begin{equation}
        \begin{split}
            \sum d_\mu d_{\gamma+\mu}
              = & \sum\limits_{r \leq i} \sum\limits_{\mu \in H_r^F} d_\mu
                + \sum\limits_{\mu \in \widetilde H_i^F} d_\mu d_{\gamma + \mu} \\
              = & \sum\limits_{r\leq i} 2^{m-r} \xi_r^2
                + (2^{m-i}-2) \xi_{i+1}^2 + 2\xi_{i+1}((-1)^i 2^m + \xi_{i+1}) \\
              = & -v.
        \end{split}
    \end{equation}
    
    Lastly, by inserting $d_\mu = \xi_r$ into $n_\mu=\frac{k+d_\mu \sqrt k}v$ we can observe that $n_\mu$ is integer iff $v | \sqrt k$.
\end{proof}

\begin{remark}
    We could also allow the value $i=0$ in Theorem \ref{z2i:theorem:construction}. In that case the obtained difference multiset would be the one described in 
Theorem \ref{regular:theorem:regular} and it would also produce  integer multiplicities for $k \equiv 1 \mod v$ or $k \equiv -1 \mod v$.
\end{remark}

\begin{theorem}
    There are no other difference multisets than those presented in Theorem \ref{z2i:theorem:construction} and Theorem \ref{regular:theorem:regular} for $m < 4$.
\end{theorem}

\begin{proof}
    This result was found by solving equations \eqref{apparatus:eq:di} and \eqref{apparatus:eq:dsystem}.
    
    For $\mathbb Z_2$ it has already been shown before \cite{buratti1999old}.
    
    We solved the $\mathbb Z_2 \times \mathbb Z_2$ and $\mathbb Z_2^3$ cases with the help of computer algebra system.
    
    For $\mathbb Z_2 \times \mathbb Z_2$ equations \eqref{apparatus:eq:di} and \eqref{apparatus:eq:dsystem} take the following form:
    
    \begin{equation}
        \begin{cases}
            d_{00} + d_{01} + d_{10} + d_{11} = 0 \\
            d_{00}^2 + d_{01}^2 + d_{10}^2 + d_{11}^2 = 12 \\
            2 d_{00}d_{01} + 2 d_{10}d_{11} = -4 \\
            2 d_{00}d_{10} + 2 d_{01}d_{11} = -4 \\
            2 d_{00}d_{11} + 2 d_{01}d_{10} = -4 \\
        \end{cases}
    \end{equation}
    
    For $\mathbb Z_2^3$ equations \eqref{apparatus:eq:di} and \eqref{apparatus:eq:dsystem} take the following form:
    
    \begin{equation}
        \begin{cases}
            d_{000} + d_{001} + d_{010} + d_{011} d_{100} + d_{101} + d_{110} + d_{111} = 0 \\
            d_{000}^2 + d_{001}^2 + d_{010}^2 + d_{011}^2 d_{100}^2 + d_{101}^2 + d_{110}^2 + d_{111}^2 = 56 \\
            2 d_{000}d_{001} + 2 d_{010}d_{011} + 2 d_{100}d_{101} + 2 d_{110}d_{111} = -8 \\
            2 d_{000}d_{010} + 2 d_{001}d_{011} + 2 d_{100}d_{110} + 2 d_{101}d_{111} = -8 \\
            2 d_{000}d_{011} + 2 d_{001}d_{010} + 2 d_{100}d_{111} + 2 d_{101}d_{110} = -8 \\
            2 d_{000}d_{101} + 2 d_{001}d_{100} + 2 d_{010}d_{111} + 2 d_{011}d_{110} = -8 \\
            2 d_{000}d_{110} + 2 d_{001}d_{111} + 2 d_{010}d_{100} + 2 d_{011}d_{101} = -8 \\
            2 d_{000}d_{111} + 2 d_{001}d_{110} + 2 d_{010}d_{101} + 2 d_{011}d_{100} = -8 \\
        \end{cases}
    \end{equation}
    
    The only solutions to these systems are the ones produced in Theorem \ref{z2i:theorem:construction} and Theorem \ref{regular:theorem:regular}.
\end{proof}


Sekojošais (līdz nodaļas beigām) jāpārdomā: vai vispār lietderīgi iekļaut. Un, ja ir, tad jāpārlabo.

The construction provided in this section produces
\begin{equation}
    2 \sum\limits_{j=2}^m 2^j \prod\limits_{l=j+1}^m (2^{l+1}-2)
\end{equation}
solutions for $d_\mu$ over $\mathbb Z_2^m$. The counting argument is that we can select $H_1$ in $2^{m+1}-2$ ways, $H_2$ in $2^m-2$ etc. until you stop and choose one of the remaining $2^j$ elements. And twice everything as you can flip the signs. Beware that taking $i=m-1$ will reproduce the difference multisets that can be obtained by taking $i=m-2$.

\begin{example}
    Let us take $m=4$ and $i=2$. We get half the digressions equal to $\xi_1=-1$. Set another quarter---four digressions to $\xi_2=3$. Out of the last four we assign one $v+\xi_3=11$ and $\xi_3=-5$ for the other three.
    
    Let us now take $m=4$ and $i=3$ instead. The first twelve digressions are assigned in the same way. Half of the remaining four are set to $\xi_3=-5$. Then we select one out of the remaining subset to set it to $-v+\xi_4=-5$ and all the remaining (i.e.\ one) to $=\xi_4=11$. Thus we end up with the same set of digressions as in the previous case.
\end{example}
