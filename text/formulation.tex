%\begin{definition}
%    \label{dms:def:ms}
%    Given a quasigroup $Q$, a multiset $M$ is called a $(Q,k)$-multiset if
%    \begin{equation}
%        |M| = k \land \forall \gamma \in M \colon \gamma \in Q
%    \end{equation}
%\end{definition}

When describing large or arbitrary multisets over a fixed quasigroup, it's convenient to use the multiplicity function $n$: the number of instances an element $\mu$ is found in a multiset $M$ will be denoted as $n(\mu,M)$ or simply $n_\mu$ if the multiset is obvious from the context.

Let $Q$ be a quasigroup.
We will use the multiplicative notation $\alpha\beta^{-1}$ for the differences in this general setting.
%In this paper we investigate the existence of such %$(Q,k)$-
%multisets that the differences of their elements produce the elements of $Q$. 
Let $M=\set{\mu_1, \mu_2, \ldots, \mu_k}$ be a multiset over $Q$. 
%we only consider the differences $\mu_i-\mu_j$ where $i \neq j$ excluding the differences between an element and itself. 
Let us denote by $\mathcal D(M)$ the multiset generated by the differences of elements of $M$ with different indices: 
$\mathcal D (M) = \set{\mu_i\mu_j^{-1} \mid i, j \in \{1,2,\ldots,k\} \land i \neq j}$.

\begin{definition}
    \label{dms:def:dc}
    A multiset $M$ of cardinality $k$ is called a $(Q,k)$-difference cover iff $\forall \gamma \in Q \colon \gamma \in \mathcal D(M)$.
\end{definition}

In particular we are interested in difference covers 
for which $\mathcal D(M)$ is regular: it contains each element of $Q$ the same number of times.
%that produce a regular multiset by the aforementioned subtractions: such that each of the $Q$ elements is produced the same number of times.

\begin{definition}
    \label{dms:def:dms}
    A $(Q,k)$-difference cover $M$ is called a $(Q,k)$-difference multiset (a.k.a.\ regular difference cover) if $\exists \lambda  \forall \gamma \in Q \colon \lambda = n(\gamma, \mathcal D(M))$.
\end{definition}

The use of symbols $\lambda$ and $k$ is consistent with their roles as parameters of the common difference sets. They serve the same purpose here and we will also use the classic notation $v = |Q|$. Commonly a $(Q,k)$-difference multiset would be called a $(Q,k,\lambda)$ (or $(v,k,\lambda)$) difference multiset, but we omit $\lambda$ as it is a function of $v$ and $k$ 
(see identity \eqref{apparatus:eq:parameters} below).

\begin{remark}
    \label{dms:remark:abelian}
%    We will use additive terminology and notation in this article because of esthetics and tradition in the field (some articles sadly tend to limit the definitions to additive groups only). Unless stated otherwise, the definitions and results apply in any group-like algebra where the inverse operation (subtraction) is possible, i.e. we're working in a quasigroup.
    Some authors, e.g.\ \cite{haanpaa2004minimum} have also investigated sum covers of groups which produce the elements of a group as sums instead of differences. However, Cayley table of sums in one quasigroup is a table of differences in another and vice versa. Thus, by generalizing to quasigroups we can treat also the sum multisets as difference multisets.
\end{remark}

\subsection{The mathematical apparatus}
    First, note that the cardinality of $(Q,k)$-difference multiset is equal to the total of multiplicities. We will omit the summation index and bounds where they are clear from the context. Suppose that all sums are over $\mu \in Q$ unless stated otherwise.
    \begin{equation}
        \label{apparatus:eq:ni}
        \sum {n_\mu} = k.
    \end{equation}
    
    Now let us restate definition \ref{dms:def:dms} in terms of $n$. Each element $\gamma$ must appear $\lambda$ times as a difference $(\gamma\mu)\mu^{-1}$. For non-identity $\gamma$ we obtain the number of $\gamma$'s occurrences by multiplying the multiplicities $n_{\gamma\mu} n_\mu$ and summing over $\mu \in Q$. For the identity we will use Kronecker delta to omit the trivial differences (i.e.\ $\mu_i\mu_i^{-1}$ where $M=\{\mu_1,\ldots,\mu_k\}$ is the multiset):
    \begin{equation}
        \label{apparatus:eq:system}
        \forall \gamma \in Q \colon \sum (n_\mu(n_{\gamma\mu}-\delta_{\mu,\gamma\mu})) = \lambda.
    \end{equation}
    
    We can observe that the number of non-trivial differences is equal to the number of $(Q,k)$-difference multiset element pairs (sub-multisets of order $2$) $k(k-1)$ and it's required to contain each of the $v=|Q|$ elements $\lambda$ times \cite{buratti1999old}:
    \begin{equation}
        \label{apparatus:eq:parameters}
        v\lambda = k(k-1).
    \end{equation}
(As is well known, a similar identity holds for the common difference sets.)
    
    These equations serve as the main tools in our investigation. Finding a $(Q,k)$-difference multiset is the same as finding a set of non-negative integer $n_\mu$'s that satisfy the above equations. 
    
    It is useful to notice that equation \eqref{apparatus:eq:ni} defines a hyperplane and equations \eqref{apparatus:eq:system} define second-order surfaces. Under this interpretation we are looking for lattice points on the intersection of all the surfaces defined by these equations.

\subsection{Digressions}
\label{sec:digressions}
    By applying the substitution $n_\mu=\frac{k+d_\mu \sqrt k}v$ we can rewrite the previous equations in terms of digressions $d_\mu$:
    
    \begin{equation}
        \label{apparatus:eq:di}
        \sum {d_\mu} = 0
    \end{equation}
    
    \begin{equation}
        \label{apparatus:eq:dsystem_general}
        \forall \gamma \in Q \colon \sum (d_\mu (d_{\gamma\mu}-\frac{v\delta_{\mu,\gamma\mu}}{\sqrt k})-v\delta_{\mu,\gamma\mu}) = -v
    \end{equation}
    
    This transformation is especially useful in the case of loops where the latter equation simplifies even further and no longer depends on $k$:
    
    \begin{equation}
        \label{apparatus:eq:dsystem}
        \forall \gamma \in Q \colon \sum d_\mu d_{\gamma\mu} = v (v \delta_{\gamma,1}-1).
    \end{equation}
    
    Thus it is a bit simpler to find a solution in terms of $d_\mu$, but the cost is that we must afterwards test if the solution produces integer $n_\mu$.
    
