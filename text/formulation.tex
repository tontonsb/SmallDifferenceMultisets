When describing large or arbitrary multisets over a fixed group, it's convenient to use the multiplicity function $n$: the number of instances an element $\mu$ is found in a multiset $M$ will be denoted as $n(\mu,M)$ or simply $n_\mu$ if the multiset is obvious from the context.

Let $G$ be an additive commutative group and $M=\set{\mu_1, \mu_2, \ldots, \mu_k}$ be a multiset over $G$. Let us denote by $\mathcal D(M)$ the multiset generated by the differences of elements of $M$ with different indices: 
$\mathcal D (M) = \set{\mu_i-\mu_j \mid i, j \in \{1,2,\ldots,k\} \land i \neq j}$.

\begin{definition}
    \label{dms:def:dc}
    A multiset $M$ of cardinality $k$ is called a $(G,k)$-difference cover iff $\forall \gamma \in G \colon \gamma \in \mathcal D(M)$.
\end{definition}

In particular we are interested in difference covers 
for which $\mathcal D(M)$ is regular: it contains each element of $G$ the same number of times.

\begin{definition}
    \label{dms:def:dms}
    A $(G,k)$-difference cover $M$ is called a $(G,k)$-difference multiset (a.k.a.\ regular difference cover) if $\exists \lambda  \forall \gamma \in G \colon \lambda = n(\gamma, \mathcal D(M))$.
\end{definition}

The use of symbols $\lambda$ and $k$ is consistent with their roles as parameters of the common difference sets. They serve the same purpose here and we will also use the classic notation $v = |G|$. Commonly a $(G,k)$-difference multiset would be called a $(G,k,\lambda)$ (or $(v,k,\lambda)$) difference multiset, but we omit $\lambda$ as it is a function of $v$ and $k$ 
(see identity \eqref{apparatus:eq:parameters} below).


\subsection{The mathematical apparatus}
    \label{sec:apparatus}
    First, note that the cardinality of $(G,k)$-difference multiset is equal to the total of multiplicities. We will omit the summation index and bounds where they are clear from the context. Suppose that all sums are over $\mu \in G$ unless stated otherwise.
    \begin{equation}
        \label{apparatus:eq:ni}
        \sum {n_\mu} = k.
    \end{equation}
    
    Now let us restate definition \ref{dms:def:dms} in terms of $n$. Each element $\gamma$ must appear $\lambda$ times as a difference $(\gamma+\mu)-\mu$. For non-identity $\gamma$ we obtain the number of $\gamma$'s occurrences by multiplying the multiplicities $n_{\gamma+\mu} n_\mu$ and summing over $\mu \in G$. For the identity we will use Kronecker delta to omit the trivial differences (i.e.\ $\mu_i-\mu_i$ where $M=\{\mu_1,\ldots,\mu_k\}$ is the multiset):
    \begin{equation}
        \label{apparatus:eq:system}
        \forall \gamma \in G \colon \sum (n_\mu(n_{\gamma+\mu}-\delta_{\gamma0})) = \lambda.
    \end{equation}
    
    We can observe that the number of non-trivial differences is equal to the number of $(G,k)$-difference multiset element pairs (sub-multisets of order $2$) $k(k-1)$ and it's required to contain each of the $v=|G|$ elements $\lambda$ times \cite{buratti1999old}:
    \begin{equation}
        \label{apparatus:eq:parameters}
        v\lambda = k(k-1).
    \end{equation}
(As is well known, a similar identity holds for the common difference sets.)
    
    These equations serve as the main tools in our investigation. Finding a $(G,k)$-difference multiset is the same as finding a set of non-negative integer $n_\mu$'s that satisfy the above equations. 
    
    It is useful to notice that equation \eqref{apparatus:eq:ni} defines a hyperplane and equations \eqref{apparatus:eq:system} define second-order surfaces. Under this interpretation we are looking for lattice points on the intersection of all the surfaces defined by these equations.

\subsection{Digressions}
\label{sec:digressions}
    By applying the substitution $n_\mu=\frac{k+d_\mu \sqrt k}v$ we can rewrite the previous equations in terms of digressions $d_\mu$:
    
    \begin{equation}
        \label{apparatus:eq:di}
        \sum {d_\mu} = 0
    \end{equation}
    
    \begin{equation}
        \label{apparatus:eq:dsystem}
        \forall \gamma \in G \colon \sum d_\mu d_{\gamma+\mu} = v (v \delta_{\gamma0}-1).
    \end{equation}
    
    This makes it a bit simpler to find a solution in terms of $d_\mu$, but the cost is that we must afterwards test if the solution produces integer $n_\mu$.
    
