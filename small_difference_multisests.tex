\documentclass{amsart}

\usepackage[foot]{amsaddr}
\usepackage{amsmath}
\usepackage{amssymb}
\usepackage{amsthm}

\usepackage{braket}  % provides \set{ | }

\usepackage{graphicx}
\usepackage{subcaption}
\usepackage{booktabs}

\usepackage{hyperref}
\usepackage{xcolor}
\hypersetup{
    colorlinks,
    linkcolor={red!50!black},
    citecolor={blue!50!black},
    urlcolor={blue!80!black}
}

\theoremstyle{plain}
\newtheorem{theorem}{Theorem}[section]
\newtheorem{lemma}[theorem]{Lemma}
\newtheorem{proposition}[theorem]{Proposition}

\theoremstyle{definition}
\newtheorem{definition}[theorem]{Definition}
\newtheorem{example}[theorem]{Example}

\theoremstyle{remark}
\newtheorem{remark}[theorem]{Remark}

\begin{document}
	\title{Difference multisets}

	\author{Juris Evertovskis}
	\email{juris.evertovskis@lu.lv}

	\author{Juris Smotrovs}
	\email{juris.smotrovs@lu.lv}
	
	\address{Faculty of Computing, University of Latvia}

	\begin{abstract}
		Difference multiset is a combinatorial design---a multiset $M$ over a 
		group $G$ such that the differences of elements of $M$ produce every 
		element of $G$ in the same number of times. In this paper we obtain 
		multiple new constructions of difference multisets as well as some 
		constraints on structure of difference multisets. We paid special 
		attention to a few cases with small parameter values and managed to 
		find all of the difference multisets over some smaller algebraic 
		structures, e.g.\ $\mathbb Z_3, \mathbb Z_2^2$ and $\mathbb Z_2^3$. 
		It turned out that there is a link between difference multisets over 
		$\mathbb Z_3$ and Löschian numbers.

		\smallskip
		\noindent \textbf{\keywordsname.} Difference multisets, Difference covers, Löschian numbers
	\end{abstract}

	\maketitle
    
    \section{Introduction}
    \subsection{Difference multisets}

    Difference multiset is a combinatorial design similar to difference set. But a multiset. 

    The classical difference set $D \subseteq G$ is such a set that produces every non-zero $\gamma \in G$ the same number of times when taking the differences between elements of $D$. A simple example is $\set{0,1} \subset \mathbb Z_3$ as $1-0=1$ and $0-1=2$ thus producing both of the non-zero elements of $\mathbb Z_3$. Curiosly, the same pair is also a difference set in $\mathbb Z_2$, but that's boring as $G$ is always a difference set of $G$. A bit less trivial and more famous example is $\set{0,1,3} \subset \mathbb Z_7$.

    If we take a multiset instead, we can produce the whole $G$, including the identity. For example, considering the differences between elements of $\set{0,0,1} \subset \mathbb Z_3$ we obtain $\set{0,0,1,1,2,2}$. This is what we call a difference multiset. Take note that we take differences from a pair of elements not an element and itself (i.e. there was $0-0=0$ as first zero subtracted from the second and vice verse but not first zero from itself and no $1-1$).

    While difference sets have been studied at least since 1939 \cite{bose1939construction}, difference multisets were first studied on their own in 1999 by Buratti \cite{buratti1999old} who noticed that such designs (and the related strong difference families) are indirectly used by other authors in constructions of various combinatorial designs. The paper defined the concept of difference multiset and obtained some theorems and constructions. The topic was developed further and renamed to regular difference covers by other authors \cite{arasu2005cyclic, arasu2005regular} who introduced new constructions and a notable amount of nonexistence theorems.

    The results in the foundational articles are mostly analogous to those of difference sets, almost all of the constructions are based on some difference set construction. As a result the number of constructed difference multisets is proportional to that of difference sets which is unlikely to reflect the real situation as there are infinitely more multisets over a given finite $G$ than there are subsets. Some constructions producing difference multisets of arbitrary size over fixed $G$ were uncovered in \cite{momihara2009strong} and we strive to expand in this direction---constructing arbitrarily large multisets in a fixed, mainly small algebras.

\subsection{Synopsis}

    We study the difference multisets using a system of mostly (and at most) quadratic equations on the multiplicities of their elements. We show that these multiplicities of any difference multiset over a loop are in a sense close to their average. This leads to the next idea of studying their digressions from the average which allows describing difference multisets with a simpler equation system.

    Using these tools we find a construction that allows us to make infinite (but not very dense) amount of difference multisets over any quasigroup. Focusing on groups $\mathbb Z_2^i$ we got a hold of a more dense construction that not only produces infinitely many difference multisets, but also produce every multiset there is for $\mathbb Z_2^i$ for at least $i \leq 3$ (we are not sure about larger values of $i$ but we suspect there are some more difference multisets there).

    We also managed to solve the problem of difference multisets of quasigroups of cardinality 3. An interesting link is found---the possible sizes of difference multisets over $\mathbb Z_3$ are Löschian numbers.
    

     
	\section{Definitions, notation and statement of the problem}
    %\begin{definition}
%    \label{dms:def:ms}
%    Given a quasigroup $Q$, a multiset $M$ is called a $(Q,k)$-multiset if
%    \begin{equation}
%        |M| = k \land \forall \gamma \in M \colon \gamma \in Q
%    \end{equation}
%\end{definition}

When describing large or arbitrary multisets over a fixed quasigroup, it's convenient to use the multiplicity function $n$: the number of instances an element $\mu$ is found in a multiset $M$ will be denoted as $n(\mu,M)$ or simply $n_\mu$ if the multiset is obvious from the context.

Let $Q$ be a quasigroup.
We will use the multiplicative notation $\alpha\beta^{-1}$ for the differences in this general setting.
%In this paper we investigate the existence of such %$(Q,k)$-
%multisets that the differences of their elements produce the elements of $Q$. 
Let $M=\set{\mu_1, \mu_2, \ldots, \mu_k}$ be a multiset over $Q$. 
%we only consider the differences $\mu_i-\mu_j$ where $i \neq j$ excluding the differences between an element and itself. 
Let us denote by $\mathcal D(M)$ the multiset generated by the differences of elements of $M$ with different indices: 
$\mathcal D (M) = \set{\mu_i\mu_j^{-1} \mid i, j \in \{1,2,\ldots,k\} \land i \neq j}$.

\begin{definition}
    \label{dms:def:dc}
    A multiset $M$ of cardinality $k$ is called a $(Q,k)$-difference cover iff $\forall \gamma \in Q \colon \gamma \in \mathcal D(M)$.
\end{definition}

In particular we are interested in difference covers 
for which $\mathcal D(M)$ is regular: it contains each element of $Q$ the same number of times.
%that produce a regular multiset by the aforementioned subtractions: such that each of the $Q$ elements is produced the same number of times.

\begin{definition}
    \label{dms:def:dms}
    A $(Q,k)$-difference cover $M$ is called a $(Q,k)$-difference multiset (a.k.a.\ regular difference cover) if $\exists \lambda  \forall \gamma \in Q \colon \lambda = n(\gamma, \mathcal D(M))$.
\end{definition}

The use of symbols $\lambda$ and $k$ is consistent with their roles as parameters of the common difference sets. They serve the same purpose here and we will also use the classic notation $v = |Q|$. Commonly a $(Q,k)$-difference multiset would be called a $(Q,k,\lambda)$ (or $(v,k,\lambda)$) difference multiset, but we omit $\lambda$ as it is a function of $v$ and $k$ 
(see identity \eqref{apparatus:eq:parameters} below).

\begin{remark}
    \label{dms:remark:abelian}
%    We will use additive terminology and notation in this article because of esthetics and tradition in the field (some articles sadly tend to limit the definitions to additive groups only). Unless stated otherwise, the definitions and results apply in any group-like algebra where the inverse operation (subtraction) is possible, i.e. we're working in a quasigroup.
    Some authors, e.g.\ \cite{haanpaa2004minimum} have also investigated sum covers of groups which produce the elements of a group as sums instead of differences. However, Cayley table of sums in one quasigroup is a table of differences in another and vice versa. Thus, by generalizing to quasigroups we can treat also the sum multisets as difference multisets.
\end{remark}

\subsection{The mathematical apparatus}
    First, note that the cardinality of $(Q,k)$-difference multiset is equal to the total of multiplicities. We will omit the summation index and bounds where they are clear from the context. Suppose that all sums are over $\mu \in Q$ unless stated otherwise.
    \begin{equation}
        \label{apparatus:eq:ni}
        \sum {n_\mu} = k.
    \end{equation}
    
    Now let us restate definition \ref{dms:def:dms} in terms of $n$. Each element $\gamma$ must appear $\lambda$ times as a difference $(\gamma\mu)\mu^{-1}$. For non-identity $\gamma$ we obtain the number of $\gamma$'s occurrences by multiplying the multiplicities $n_{\gamma\mu} n_\mu$ and summing over $\mu \in Q$. For the identity we will use Kronecker delta to omit the trivial differences (i.e.\ $\mu_i\mu_i^{-1}$ where $M=\{\mu_1,\ldots,\mu_k\}$ is the multiset):
    \begin{equation}
        \label{apparatus:eq:system}
        \forall \gamma \in Q \colon \sum (n_\mu(n_{\gamma\mu}-\delta_{\mu,\gamma\mu})) = \lambda.
    \end{equation}
    
    We can observe that the number of non-trivial differences is equal to the number of $(Q,k)$-difference multiset element pairs (sub-multisets of order $2$) $k(k-1)$ and it's required to contain each of the $v=|Q|$ elements $\lambda$ times \cite{buratti1999old}:
    \begin{equation}
        \label{apparatus:eq:parameters}
        v\lambda = k(k-1).
    \end{equation}
(As is well known, a similar identity holds for the common difference sets.)
    
    These equations serve as the main tools in our investigation. Finding a $(Q,k)$-difference multiset is the same as finding a set of non-negative integer $n_\mu$'s that satisfy the above equations. 
    
    It is useful to notice that equation \eqref{apparatus:eq:ni} defines a hyperplane and equations \eqref{apparatus:eq:system} define second-order surfaces. Under this interpretation we are looking for lattice points on the intersection of all the surfaces defined by these equations.

\subsection{Digressions}
\label{sec:digressions}
    By applying the substitution $n_\mu=\frac{k+d_\mu \sqrt k}v$ we can rewrite the previous equations in terms of digressions $d_\mu$:
    
    \begin{equation}
        \label{apparatus:eq:di}
        \sum {d_\mu} = 0
    \end{equation}
    
    \begin{equation}
        \label{apparatus:eq:dsystem_general}
        \forall \gamma \in Q \colon \sum (d_\mu (d_{\gamma\mu}-\frac{v\delta_{\mu,\gamma\mu}}{\sqrt k})-v\delta_{\mu,\gamma\mu}) = -v
    \end{equation}
    
    This transformation is especially useful in the case of loops where the latter equation simplifies even further and no longer depends on $k$:
    
    \begin{equation}
        \label{apparatus:eq:dsystem}
        \forall \gamma \in Q \colon \sum d_\mu d_{\gamma\mu} = v (v \delta_{\gamma,1}-1).
    \end{equation}
    
    Thus it is a bit simpler to find a solution in terms of $d_\mu$, but the cost is that we must afterwards test if the solution produces integer $n_\mu$.
    

	
    \section{Main results}
    \subsection{Limits for multiplicities}
    Considering difference multisets over an arbitrary loop (quasigroup with an identity) one can notice that some of the surfaces defined by our equations are always the same. There is always the hyperplane $\sum {n_\mu} = k$ and the hypersphere $\sum n_\mu^2 = k + \lambda$ centered at the origin (see Figure \ref{general:figure:surfaces}). 
%One could notice right away that 
The second equation confines every multiplicity: $n_\mu \leq \sqrt{k+\lambda}$. By investigating the intersection more thoroughly we discover that the multiplicities are actually bound to be near to their average $k/v$.

    \begin{figure}
        \centering
        \begin{subfigure}[b]{0.5\textwidth}
            \includegraphics[width=\textwidth]{assets/surfacesIn2D}
        \end{subfigure}%
        ~
        \begin{subfigure}[b]{0.5\textwidth}
            \includegraphics[width=\textwidth]{assets/surfacesIn3D}
        \end{subfigure}
        \caption{The $\sum {n_\mu} = k$ and $\sum n_\mu^2 = k + \lambda$ surfaces in two and three dimensions.}
        \label{general:figure:surfaces}
    \end{figure}
        
    \begin{theorem}
        \label{general:theorem:limits}
        If $M$ is a $(Q,k)$-difference multiset and $|Q|=v$ then
        \begin{equation}
            \forall \gamma \in Q \colon\qquad \frac{k-(v-1)\sqrt k}{v} \leq n(\gamma,M) \leq \frac{k+(v-1)\sqrt k}{v}
        \end{equation}
    \end{theorem}
    
    \begin{proof}
        Take equation \eqref{apparatus:eq:system} for the identity element and equation \eqref{apparatus:eq:ni} as constraints:
        
        \begin{equation}
            \begin{cases}
                \sum {n_\mu} = k \\
                \sum (n_\mu(n_\mu-1)) = \lambda.
            \end{cases}
        \end{equation}
        
        Let us optimize $n_\gamma=n(\gamma,M)$ respecting the constraints. Add the first equation to the second and move all the terms to the left hand side:
        
        \begin{equation}
            \begin{cases}
                k - \sum {n_\mu} = 0 \\
                k + \lambda - \sum n_\mu^2 = 0.
            \end{cases}
        \end{equation}
        
        We apply the method of Lagrange multipliers to obtain the maximum and minimum of $n_\gamma$. 
%honoring the constraints 
We use the following Lagrange function ($\lambda_1$ and $\lambda_2$ here is the standard Lagrange multiplier notation and have nothing in common with the parameter $\lambda$).
        
        \begin{equation}
            \mathcal L = n_\gamma - \lambda_1 (k - \sum n_\mu) - \lambda_2 (k + \lambda - \sum n_\mu^2)
        \end{equation}
        
%        This gives the stated boundaries for any $n_\gamma$ in a difference multiset. The optimization calculations are not included as those are tedious and in no way novel.
        By a standard application of this method the bounds of the theorem statement are obtained. We omit the routine calculations.
    \end{proof}

    Theorem \ref{general:theorem:limits} 
%might appear uninspiring at first but it not only 
suggests using digressions instead of multiplicities (thus simplifying the equations) and greatly reduces the amount of options for every $n_\gamma$. This simplification allows decent computer searches which enabled us to discover some of the patterns that led to results presented in this paper.
        
    \begin{figure}
        \includegraphics[width=\textwidth]{assets/boundingSurfaces}
        \caption{Lower and upper bounds for the values of $n_\gamma$ with respect to $v$ and $k$.}
        \label{general:figure:limits}
    \end{figure}
    
\subsection{A family of difference multisets for every quasigroup}
%    Based on particular results discussed further, we have discovered a construction that works whenever $k$ is close to a multiple of $v$.
    
    \begin{theorem}
        \label{regular:theorem:regular}
        If $\sqrt k$ is integer and congruent to $0$ or $\pm 1 \mod v$ then a $(Q,k)$-difference multiset exists with the following digressions.
            \begin{itemize}
                \item If $\sqrt k \equiv 1 \mod v$ then $d_\mu = v-1$ for any single element $\mu$ and $d_{\nu \neq \mu} = -1$ for the other elements.
                \item If $\sqrt k \equiv -1 \mod v$ then $d_\mu =1-v$ for any single element $\mu$ and $d_{\nu \neq \mu} = 1$ for the other elements.
                \item Both of the above constructions if $\sqrt k \equiv 0 \mod v$.
            \end{itemize}
    \end{theorem}
    
    \begin{proof}
        The conditions on the values of $k$ and $d_\mu$ guarantee that the multiplicities $n_\mu=\frac{k+d_\mu \sqrt k}v$ are integers. It is left to demonstrate that they form a difference multiset.
        
        Considering equation \eqref{apparatus:eq:dsystem} for non-identity elements we can notice that the digression $\pm(v-1)$ (as any other digression) is involved in two of the products---once as $d_\mu$ and once as $d_{\gamma\mu}$.
% (but not both at the same time as we consider non-identity $\gamma$ now). 
Other $v-2$ products are $1\cdot 1$ or $(-1)\cdot(-1)$, thus the condition is satisfied:
        
        \begin{equation}
            \sum d_\mu d_{\gamma\mu} = -2(v-1) + (v-2) = -v.
        \end{equation}

        In the case of a loop, the condition for the identity $\gamma$ is also satisfied:
        
        \begin{equation}
            \sum d_\mu^2  = \left( \pm (v-1) \right)^2 + (v-1) \left( \mp 1 \right)^2 = v^2 - v.
        \end{equation}
    
        It is straightforward to check that equation \eqref{apparatus:eq:di} is satisfied as well.
    \end{proof}

\subsection{Difference multisets for cyclic groups}
    For cyclic groups we can write \eqref{apparatus:eq:dsystem} as
    \begin{equation}
    \label{general:eq:matrix_eq}
        D d = \bf v
    \end{equation}\
    where $d = (d_0, d_1, \ldots, d_{v-1})$, $D_{\mu\nu} = d_{\mu+\nu}$ and $\bf v = (v^2-v, -v, -v, \ldots)$.
    
    The form of the matrix $D$ is the following:
    \begin{equation}
        \label{general:eq:anticirculant_matrix}
        D =
        \begin{pmatrix}
            d_0 & d_1 & d_2 & \cdots & d_{v-1} \\ 
            d_1 & d_2 & d_3 & \cdots & d_0 \\
            d_2 & d_3 & d_4 & \cdots & d_1 \\
            \vdots & \vdots & \vdots & \ddots & \vdots \\
            d_{v-2} & d_{v-1} & d_0 & \cdots & d_{v-3} \\
            d_{v-1} & d_0 & d_1 & \cdots & d_{v-2} \\
        \end{pmatrix}
    \end{equation}
    
    This is a special case of Hankel matrix sometimes called \emph{anticirculant matrix}. The structure of the corresponding $D$ will be the same.
    
    \subsubsection{Solving equations with anticirculant matrices}
    \label{general:sec:anticirculant}
%        We weren't able to track down a source dealing with matrices like \eqref{general:eq:anticirculant_matrix}, however we managed to apply the same methods as with circulant matrices \cite{wiki:circulant_matrix}.
          To solve \eqref{general:eq:matrix_eq} we apply similar methods to those used to solve systems of linear equations with circulant matrices, see e.~g.\ \cite{wiki:circulant_matrix}.
        
        Let us consider the equation $Ax=b$ with an anticirculant $v \times v$ matrix
        \begin{equation}
            A =
            \begin{pmatrix}
                a_0 & a_1 & a_2 & \cdots & a_{v-1} \\ 
                a_1 & a_2 & a_3 & \cdots & a_0 \\
                a_2 & a_3 & a_4 & \cdots & a_1 \\
                \vdots & \vdots & \vdots & \ddots & \vdots \\
            \end{pmatrix}
        \end{equation}\
        and a vector $b=(b_0, b_1, \ldots, b_{v-1})^T$. 
        
        Denote $a=(a_0,a_1,\ldots,a_{v-1})$. We can now express the equation row by row (please consider indices $\mod v$):
        \begin{equation}
            b_j = \sum_{i=0}^{v-1} a_{j+i} x_i
        \end{equation}
        
        Apply discrete Fourier transform:
        \begin{equation}
            \mathcal{F} (b)_m = \sum_{j=0}^{v-1} b_j \omega^{-jm}
        \end{equation}\
        where $\omega = \exp(\frac{2\pi i}v)$---a root of unity (this is the only place where $i$ is used to denote the imaginary unit).
        
        Insert $b_j$ obtaining
        \begin{equation}
        \begin{split}
            \mathcal{F} (b)_m
            &= \sum_{j=0}^{v-1} \sum_{i=0}^{v-1} a_{j+i} x_k \omega^{-jm} \\
            &= \sum_{j=0}^{v-1} \sum_{i=0}^{v-1} x_i \omega^{im} a_{j+i} \omega^{-(j+i)m}  \\
            &= \sum_{i=0}^{v-1} x_i \omega^{im} \sum_{j'=i}^{v+i-1} a_{j'} \omega^{-j'm}
        \end{split}
        \end{equation}
        
        Looking at the inner sum we should note that not only we take indices $\mod v$ but the $j'$ in the exponent can be taken $\mod v$ as well. Let's use $j'' = j' \mod v$.
        \begin{equation}
            \sum_{j'=i}^{v+i-1} a_{j'} \omega^{-j'm} = 
            \sum_{j''=0}^{v-1} a_{j''} \omega^{-j''m}= \mathcal{F}(a)_m
        \end{equation}
        
        We can now finish the transformation:
        \begin{equation}
        \begin{split}
            \mathcal{F} (b)_m
            &= \sum_{i=0}^{v-1} x_i \omega^{im} \mathcal{F}(a)_m \\
            &= \mathcal{F}(a)_m (\sum_{i=0}^{v-1} x^*_i \omega^{-im})^* \\
            &= \mathcal{F}(a)_m \mathcal{F}^*(x^*)_m 
            = v \mathcal{F}(a)_m \mathcal{F}^{-1}(x)_m
        \end{split}
        \end{equation}
        
        The final form (exploiting the Fourier transform property $\mathcal{F}^{-1}(x) = \mathcal{F}^*(x^*)/v$) was included for completeness as it allows to explicitly express $x = \mathcal{F} \left(\frac1v \frac{\mathcal{F}(b)}{\mathcal{F}(a)} \right)$. However, we will use $\mathcal{F} (b)_m = \mathcal{F}(a)_m \mathcal{F}^*(x^*)_m$.
    
    \subsubsection{Solving the digression equation}
        The $D$ and $d$ in equation $Dd=\bf v$ is linked in the same way as $A$ and $a$ in section \ref{general:sec:anticirculant}. The image of $Dd=\bf v$ is
        
        \begin{equation}
            \mathcal{F} ({\bf v})_m = \mathcal{F}(d)_m \mathcal{F^*}(d^*)_m
        \end{equation}

        As we are only interested in real $d$, we can even simplify it to
        
        \begin{equation}
            \label{general:eq:dfourier}
            \mathcal{F} ({\bf v})_m = \mathcal{F}(d)_m \mathcal{F^*}(d)_m = |\mathcal{F}(d)_m|^2
        \end{equation}
        
        Remembering ${\bf v} = (v^2-v, -v, -v, \ldots)$ we can find that $\mathcal{F}({\bf v'}) = (0,v^2,v^2,\ldots)$ and \eqref{general:eq:dfourier} becomes
        \begin{equation}
            \label{gen:eq:dfourierfinal}
            \left| \sum_{\mu=0}^{v-1} d_\mu \omega^{-\mu m} \right| = v (1-\delta_{m0})
        \end{equation}

        For any $m | v$ we can note that $\delta_{m0}=0$ and
        \begin{equation}
            \label{general:eq:split_fourier}
            \left| \sum_{\mu=0}^{v-1} d_\mu \omega^{-m\mu} \right|
            = \left| \sum_{\mu=0}^{v/m-1} \sum_{\nu=0}^{m-1}  d_{\mu+\nu v/m} \omega^{-m\mu} \right|
            =v
        \end{equation}\
        as 
        \begin{equation}
            e^{\frac{-2\pi i m (\mu+\nu v/m)}v} = e^{\frac{-2\pi i m \mu}v} e^{-2\pi i \nu} = e^{\frac{-2\pi i m \mu}v}
        \end{equation}
        
        We consider expressions \eqref{gen:eq:dfourierfinal} and \eqref{general:eq:split_fourier} as the main results of this section, here's how one can use it.
        
        \begin{proposition}
            \label{general:theorem:even_cyclic}
            In cyclic groups of even cardinality $\sum_{\mu=0}^{v/2-1} d_{2\mu} = \pm \frac v2$ and $\sum_{\mu=0}^{v/2-1} d_{2\mu+1} = \mp \frac v2$.
        \end{proposition}
        \begin{proof}
            Take \eqref{general:eq:split_fourier} for $m=v/2$
            \begin{equation}
                \left| \sum_{\mu=0}^{v/2-1} d_{2\mu} - \sum_{\mu=0}^{v/2-1} d_{2\mu+1} \right| = v
            \end{equation}
            
            We've split $d$ in half and got that total of one half is by $v$ larger than the total of the other half. The statement of the theorem follows as soon as we remember the grand total $\sum d_\mu = 0$.
        \end{proof}
        
        \begin{remark}
            Similar relation also holds true for some (many? all?) other structures that are not cyclic groups. For example in $\mathbb Z_2 \times \mathbb Z_2$ with elements $\set{\mu, \nu, \zeta, \eta}$ in any order we have $d_\mu+d_\nu-(d_\zeta+d_\eta)=\pm 4$.
        \end{remark}


\subsection{Difference multisets over $\mathbb Z_2^i$}
    \label{sec:z2n}
    We obtained a construction that produces plenty of difference multisets in $\mathbb Z_2^i$. We shall start by explaining the construction and then a proof and analysis of the construction will be presented.

    \subsubsection{Construction}
        Consider the elements $\mu \in \mathbb Z_2^i$ as i-tuples $\mu=(\mu_1, \ldots, \mu_i)$.
        
        Select a hyperplane $H_1$ out of $\mathbb Z_2^i$ defined by equation $0 = a_0 + a_1 \mu_1 + a_2 \mu_2 + \ldots + a_i \mu_i$ ($0=a_0+a\cdot \mu$) with $a_\nu \neq 0$ for at least one $\nu \neq 0$. Set $d_\eta = -1$ for every $\eta \in H_1$.
        
        As for the remaining $(i-1)$-dimensional halfspace: take a hyperplane $H_2$ out of this and set $d_\eta = 3$ for every $\eta \in H_2$.
        
        Repeat this process $0 \leq m \leq i-1$ times setting $d_\eta = \sum\limits_{j=0}^k (-2)^j$ for every $\eta \in H_k$.
        
        You will end up with the final subspace $H_f$ remaining. Select an element $\gamma$ and set $d_\gamma = (-1)^m v + \sum\limits_{j=0}^{m+1} (-2)^j$. Set $d_\eta = \sum\limits_{j=0}^{m+1} (-2)^j$ for the remaining $\eta \in H_f$.
        
        One can also flip the sign on every $d_\mu$ getting another bunch of difference multisets.
        
    \subsubsection{A few examples}
        
        \begin{example}
            Take $i=7$. Thus $v=2^i=128$. Take hyperplane $H_1$ defined by $0=\mu_1$, and set $d_\eta=-1$ for all $\eta\in H_1$ i.e. set $d_{0000000}=d_{0000001}=\ldots=d_{0111111}=-1$.
            
            Let's continue with the remaining subspace ($0=1+\mu_1$). Select another halfpace $H_2$ defined by $0=\mu_2$ and set $d_{1000000}=\ldots=d_{1011111}=3$.
            
            Let's choose $m=4$. We must then repeat the bisections two more times setting $d_\eta=-5$ for $\eta\in H_3$ and $d_\eta=11$ for $\eta \in H_4$.
            
            We have 8 elements left. Let's set $d_{1111111} = (-1)^{m} v + \sum\limits_{j=0}^{m+1} (-2)^j = v - 21 = 107$ and it remains that the other $d_{1111000}=\ldots=d_{1111110}=-21$.
        \end{example}
        
        For tighter examples (with $m\geq i-2$) the multiplicity of the final element will take form of $\sum (-2)^j$ as well. All the digressions will appear to be on the sequence $-1,3,-5,11,-21,43,-85,\ldots$ \cite{A077925}.
        
        \begin{example}
          Take $i=4$ and $m=2$. You will have eight $d_\mu=-1$, four $d_\mu=3$, three $d_\mu=-5$ and one $d_\mu=11$.
        \end{example}
        
        \begin{example}
            \label{2n:example:edge}
            Let's take $i=4$ and $m=3$. You get half the digressions (eight) $d_\mu=-1$. You set another quarter---four digressions $d_\mu=3$. Then you set two $d_\mu=-5$. Halfspace with two elements remains. All except one are set to $d_\mu=11$. And the last one is $-v+11=-5$ Thus you end up with the same set of digressions as in the previous example.
        \end{example}

        Example \ref{2n:example:edge} shows that some of the constructions (the ones with $m=i-1$) produce a difference multiset that coincides with the $m=i-2$ construction.
    
    \subsubsection{Proof}
        For a selected $0 \leq m \leq i-1$ this construction provides us with $2^{i-l}$ digressions of value $d_\eta=\sum\limits_{j=0}^l(-2)^j$ for each $1<l\leq m$ (none of these if $m=0$), $2^{i-m}-1$ digressions equal to $\sum\limits_{j=0}^{m+1}(-2)^j$ and one digression equal to $(-1)^m 2^i+\sum\limits_{j=0}^{m+1}(-2)^j$.
        
        Checking equation $\sum d_\mu = 0$ and $\sum d_\mu^2 = v(v-1)$ is straightforward if you take into account that $\sum\limits_{j=0}^l(-2)^j=(1-(-2)^{l+1})/3$.
        
        Equations \eqref{apparatus:eq:dsystem} are left to check. We began the construction by selecting a hyperplane $H_1$ defined by $0=a_0+a\cdot \mu$ where $a=(a_,a_2,\ldots)$ and $\mu=(\mu_1,\mu_2,\ldots)$ Depending on selection of $\gamma=(\gamma_1,\gamma_2,\ldots)$ there are two cases:
        \begin{itemize}
            \item If $1 \equiv a\cdot \gamma \mod 2$ then $\forall \mu \in H_1 : \mu+\gamma \notin H1 $ and $\forall \mu \notin H_1 : \mu+\gamma \in H1$;
            \item If $0 \equiv a\cdot \gamma \mod 2$ then $\forall \mu \in H_1 : \mu+\gamma \in H1 $ and $\forall \mu \notin H_1 : \mu+\gamma \notin H1$.
        \end{itemize}
        
        In the first case every $d_\mu d_{\mu+\gamma}$ involves factor $-1$. $\sum d_\mu d_{\mu+\gamma} = -2\sum\limits_{i\notin H_1} d_\mu = -2 (\sum d_\mu - \frac v2 (-1)) = -v$.
        
        In the second case $d_\mu d_{\mu+\gamma} = 1$ for every $\mu \in H_1$ and $\sum\limits_{\mu \in H_1} d_\mu d_{\mu+\gamma} = \frac v2$. So the remaining stuff must make up $-\frac {3v}2$ For the remaining stuff we are once again split into two cases depending on $\gamma$ and the initial choice of $H_2$. Either both $\mu$ and $\mu+\gamma$ belong to the different sub-hyperplanes for every $\mu$, or they belong to the same for every $\mu$. That is, we either have $\sum\limits_{j=0}^2 (-2)^j = 3$ in every factor or we continue the process.
        
        In general for any $\gamma$ we will end up at some step where we will have already summed up $d_\mu^2$ for $\mu \in H_1 \cup H_2 \cup \ldots \cup H_r$ and at the next step we will have one of these cases:
        
        \begin{itemize}
            \item No more $H_{r+1}$ has been constructed---only $2^{i-r}-1$ elements with $d_\mu = \sum\limits_{j=0}^{r+1} (-2)^j$ and a single $d_\gamma = (-1)^m 2^i + \sum\limits_{j=0}^{r+1} (-2)^j$;
            \item $\mu \in H_{r+1}$ will have to be multiplied with the items outside $H_{r+1}$ (like the first case in the previous fork).
        \end{itemize}

        The first case checks out:
        \begin{equation}
            \begin{split}
                \sum d_\mu^2 & 
                = \sum\limits_{H_1 \cup \ldots \cup H_r} d_\mu^2 \\
                & + (2^{i-r}-2) \left(\sum\limits_{j=0}^{r+1} (-2)^j \right)^2 \\
                & + 2 \left(\sum\limits_{j=0}^{r+1} (-2)^j \right) \left( (-1)^r 2^i + \sum\limits_{j=0}^{r+1} (-2)^j \right) \\
                = & - 2^i = -v
            \end{split}
        \end{equation}
    
        The second does as well:
        \begin{equation}
            \begin{split}
                \sum d_\mu^2 & \\
                = & \sum\limits_{H_1 \cup \ldots \cup H_r} d_\mu^2 \\
                + & 2 \left(\sum\limits_{j=0}^{r+1} (-2)^j \right) 
                 \Bigg(
                    \sum\limits_{l=r+2}^{m} 2^{i-l} \sum\limits_{j=0}^l (-2)^j \\
                   & + (2^{i-m}-1)\sum\limits_{j=0}^{m+1} (-2)^j \\
                   & + (-1)^m 2^i + \sum\limits_{j=0}^{m+1} (-2)^j
                 \Bigg) \\
                = & - 2^i = -v
            \end{split}
        \end{equation}
    
    \subsubsection{Analysis}
        For $i \leq 3$ the construction makes all the difference multisets there are. This can be shown explicitly by solving the digression equations.
        
        We don't know about larger groups. Our construction produces $2^i$ different values for $i \leq 3$ but only 12, 16 and 20 different $d_\mu$ values for $i$ of 4,5 and 6 respectively.
        
        As for the number of difference multisets, this construction produces
        \begin{equation}
            2 \sum\limits_{j=2}^i 2^j \prod\limits_{l=j+1}^i (2^{l+1}-2)
        \end{equation}\
        solutions for $d_\mu$ over $\mathbb Z_2^i$. The counting argument is that we can select $H_1$ in $2^{i+1}-2$ ways, $H_2$ in $2^i-2$ etc. until you stop and choose one of the remaining $2^j$ elements. And twice everything as you can flip the signs.
        
        The difference multisets (i.e. integer solutions $n_\mu=\frac{k+d_\mu \sqrt k}v$) themselves are produced whenever $v | \sqrt k$. In addition the cases of single $d_\gamma = \pm (v-1)$ and the rest $d_\mu = \mp 1$ we get integer $n_\mu$ for $k \equiv \mp 1 \mod v$.
        
\subsection{Difference multisets over the three element group}
    \label{sec:z3}
    There is only one group of three elements. Let's take it in form of $\mathbb Z_3$. What must the $k$ be for $(\mathbb Z_3,k)$-difference multiset to exist? What are these difference multisets and how many of them are there for a particular value of $k$?

    To answer these questions we shall write down \eqref{apparatus:eq:system} for a non-identity element and combine it with \eqref{apparatus:eq:ni} and \eqref{apparatus:eq:parameters} to form a system of equations.

    \begin{equation}
        \label{v3:eq:constraints}
        \begin{cases}
            3\lambda = k(k-1) \\
            \sum n_\mu = k \\
            \sum n_\mu n_{\mu+1} = \lambda
        \end{cases}
    \end{equation}

    We may now combine the equations to discover a relation between multiplicities of elements.

    \begin{theorem}
        \label{v3:theorem:relations}
        Multiplicities of different $(\mathbb Z_3,k)$-difference multiset elements $\mu$ un $\nu$ are related via
        \begin{equation}
            \label{v3:eq:relations}
            n_{\mu\neq \nu} = \frac{k-n_\nu \pm \sqrt{\frac{4k-(k-3n_\nu)^2}{3}}}{2}
        \end{equation}
    \end{theorem}

    \begin{proof}
        Take any element $\gamma \in \mathbb Z_3$ and assign $c = n_\gamma$. Let's use $\alpha$ and $\beta$ to name the remaining elements of $\mathbb Z_3$. The system \eqref{v3:eq:constraints} can now be rewritten:
        \begin{equation}
            \begin{cases}
                n_\alpha + n_\beta = k - c \\
                n_\alpha n_\beta + c (n_\alpha + n_\beta)  = \lambda 
            \end{cases}
        \end{equation}
        
        Substitute $k'=k-c$ and $\lambda' = \lambda + c^2-kc$ to obtain
        
        \begin{equation}
            \begin{cases}
                n_\alpha + n_\beta = k' \\
                n_\alpha n_\beta = \lambda'
            \end{cases}
        \end{equation}
        
        Eliminating $n_\beta$ we arrive at a quadratic equation that is solved into
        
        \begin{equation}
            n_\alpha = \frac{k' \pm \sqrt{k'^2-4\lambda'}}{2}
        \end{equation}
        
        Undo the substitutions and you're done.
    \end{proof}

    Considering the multiplicities in form of $n_\mu = \frac{k+\Delta_\mu}{3}$, we can restate \eqref{v3:eq:relations} into the following.

    \begin{equation}
        \label{v3:eq:relations_delta}
        n_{\mu\neq \nu} = \frac{k-n_\nu \pm \sqrt{\frac{4k-\Delta_\nu^2}{3}}}{2}
    \end{equation}

    The rest of analysis focuses on the $\Delta_\mu$ and it's effect on the above equation. The behaviour of expression under the root is tied to a topic in number theory called Löschian numbers \cite{oeisA003136}. These numbers make an appearance in a variety of fields (see comments in \cite{oeisA003136}).

    \begin{definition}
        \label{v3:def:loeshian}
        Number $k$ is called a Löschian number if $\exists a,b \in \mathbb Z \colon a^2+ab+b^2=k$.
    \end{definition}

    For our purposes (to eliminate unnecessary symmetries) we will only consider $a,b$ such that $a \geq b \geq 0$. This, however, doesn't change the scope of Löschian numbers.

    \begin{lemma}
        \label{v3:lemma:loeschian}
        For any Löschian number $k$ we can find $a,b \in \mathbb Z$ such that $a^2+ab+b^2=k$ and $a \geq b \geq 0$.
    \end{lemma}

    \begin{proof}
        As $k$ is a Löschian number there are $a',b' \colon a'^2+a'b'+b'^2=k$. We can construct $a,b$ such that $a^2+ab+b^2=k$ and $a \geq b \geq 0$ as follows:
        \begin{itemize}
            \item If $a' \geq 0$ and $b' \geq 0$ just take $a=a'$ and $b=b'$ or swap them if $a'<b'$.
            \item If $a'<0,b'<0$ take $a'=-a,b'=-b$ or swap them if $a'>b'$.
            \item If $ab<0$ take either $a'=|a|, b'=|a+b|$ or $a'=|a+b|, b'=|b|$. Swap places as necessary to ensure $a \geq b \geq 0$.
        \end{itemize}
    \end{proof}

    Having introduced the term, we may now introduce the promised link.

    \begin{lemma}
        \label{v3:lemma:square}
        There exists a $\Delta$ that makes $\frac{4k-\Delta^2}{3}$ a perfect square iff $k$ is Löschian number.
        
        $\Delta$ values that does the job are $\pm (2a+b), \pm (a+2b), \pm (a-b)$, where $a,b$ are such that $a \geq b \geq 0$ and $a^2+ab+b^2=k$. There is no other $\Delta$ that makes $\frac{4k-\Delta^2}{3}$ into square.
    \end{lemma}

    \begin{proof}
        For a Löschian number $k=a^2+ab+b^2$ take $\Delta$ equal to $\pm (2a+b)$, $\pm (a+2b)$ or $\pm (a-b)$ and obtain the value of expression in question to be $b^2$, $a^2$ or $(a+b)^2$ which are clearly squares.
        
        On the other hand, if $\frac{4k-\Delta^2}{3}$ is square, assign:
        \begin{equation}
            z^2 = \frac{4k-\Delta^2}{3}
        \end{equation}
        
        Rewrite
        \begin{equation}
            \frac{3z^2 + \Delta^2}{4} = k
        \end{equation}
        
        Noticing that $4$ divides $3z^2 + \Delta^2$ we can conclude that $z$ and $\Delta$ are of the same parity (because $z^2 \equiv \Delta^2 \mod 4$). Thus $2$ divides both $\Delta-z$ and $\Delta+z$.
        
        We can now find integers $a,b$ such that $a \geq b \geq 0$ and $a^2+ab+b^2=k$ (thus $k$ is a Löschian number) and the $\Delta$ can be expressed in one of the expressions stated in lemma.
        
        \begin{itemize}
            \item If $z \geq \Delta$ take $a=\frac{z+\Delta}{2}$ and $b=\frac{z-\Delta}{2}$. Then $a-b=\Delta$.
            \item If $\Delta \geq z \geq \frac \Delta 3$ take $a=z$ and $b=\frac{\Delta-z}{2}$. Then $a+2b=\Delta$.
            \item If $\frac \Delta 3 \geq z$ take $a=\frac{\Delta-z}{2}$ and $b=z$. Then $2a+b=\Delta$.
        \end{itemize}
    \end{proof}

    Let's introduce the following notation for the three values used in lemma \ref{v3:lemma:square}. The rest can be expressed as $-\Delta_i$:
    \begin{equation}
        \label{v3:eq:deltas}
        \Delta_\alpha = 2a+b, \Delta_\beta = -a-2b, \Delta_\gamma = -a+b
    \end{equation}

    These $\Delta_i$ will be used in the following theorem and $\alpha$, $\beta$ and $\gamma$ are labels that, as before, we use to label the elements of $\mathbb Z_3$ in arbitrary order. We can now state our main result which is both construction and existence criterion for $(\mathbb Z_3,k)$-difference multisets.

    \begin{theorem}
        \label{v3:theorem:loeschian}
        For every pair $a,b \in \mathbb Z$ such that $k=a^2+ab+b^2$ and $a \geq b \geq 0$ there are exactly $-(k+1) \mod 3$ (up to automorphisms) $(\mathbb Z_3,k)$-difference multisets and the multiplicities of their elements are
        
        \begin{itemize}
            \item $n_\mu=\frac{k+\Delta_\mu}{3}$ for one and $n_\nu=\frac{k-\Delta_\nu}{3}$ for the other if $3 \mid k$.
            \item $n_\mu=\frac{k+\Delta_\mu}{3}$ if $3 \nmid k$ un $b-a \equiv 1 \mod 3$.
            \item $n_\mu=\frac{k-\Delta_\mu}{3}$ if $3 \nmid k$ un $a-b \equiv 1 \mod 3$.
        \end{itemize}
    \end{theorem}

    \begin{proof}
        According to lemma \ref{v3:lemma:square}, the expression \eqref{v3:eq:relations_delta} will equal integer only if $k$ is a Löschian number and $\Delta_\mu$ is one of the listed on \eqref{v3:eq:deltas} or a negative of that.
        
        Insert the constructions listed in \eqref{v3:theorem:loeschian} into \eqref{v3:eq:relations} to check that these are indeed multiplicities that make up a difference multiset if the numbers are whole. One can also check that using $\Delta_\alpha$ to construct one of the multiplicities you will find $\Delta_\beta$ and $\Delta_\gamma$ used for the others and the same is true in any order.
        
        Considering remainders one may check the following:
        \begin{itemize}
            \item If $a \equiv b \mod 3$ then $3 \mid k$ and all the multiplicities in both the constructions $n_\mu=\frac{k+\Delta_\mu}{3}$ and $n_\mu=\frac{k-\Delta_\mu}{3}$ are integers.
            \item If $a \equiv b-1 \mod 3$ then $k \equiv 1 \mod 3$ and only the multiplicities constructed by $n_\mu=\frac{k+\Delta_\mu}{3}$ are all integer.
            \item If $a \equiv b+1 \mod 3$ then $k \equiv 1 \mod 3$ and only the multiplicities constructed by $n_\mu=\frac{k-\Delta_\mu}{3}$ are all integer.
        \end{itemize}
    \end{proof}

    \begin{remark}
        Allowing $a,b$ such that $a \geq b \geq 0$ wouldn't hold, we'd obtain the same $\Delta_\alpha, \Delta_\beta, \Delta_\gamma$ in different order thus making the same difference multisets again (up to automorphism). This constraint is intended to exclude such symmetries.
        Different $a \geq b \geq 0$ pairs with $a^2+ab+b^2=k$ will lead to different value of $a-b$ and thus all the constructions mentioned in \ref{v3:theorem:loeschian} will be distinct. Consequently the number of $(\mathbb Z_3,k)$ will be proportional to number of unique $a,b$ pairs (respecting constraints) and the coefficient of proportionality is $-(k+1) \mod 3$.
    \end{remark}

    \subsection{Estimating numbers}
        Despite our effort, the exact number of solutions is still elusive. This aspect is now reduced to a number-theoretic question -- how many unique solutions are there for $k=a^2+ab+b^2$ such that $a\geq b\geq 0$.

        The number of solutions without the constraint is known \cite{marmon2005hexagonal}. Denote
        \begin{equation}
            k=3^\alpha p_1^{\alpha_1}p_2^{\alpha_2}\ldots q_1^{\beta_1}q_2^{\beta_2}\ldots
        \end{equation}\
        
        where $p_i$ are primes such that $p_i \equiv 1 \mod 3$ and $q_i$ are primes such that $q_i \equiv 2 \mod 3$. If any of the $\beta_i$ are odd, there are no integer solutions to $k=a^2+ab+b^2$. But if all of $\beta_i$ are even, the number of solutions is $6\prod (\alpha_i +1)$.
        
        It is hypothesised \cite{nair2004elementary} that the number of solutions (if every $\beta_i$ is even) having $a \geq b \geq 0$ is $1/2 + \prod (\alpha_i +1)/2$ if all the $\alpha_i$ are even and $\prod (\alpha_i +1)/2$ otherwise. We checked this to be true for a thousand Löschian numbers. However, for most of the Löschian numbers this remains unchecked.

\subsection{Other quasigroups of size 3}
    \label{sec:v3}
    As mentioned in the opening sections, one might also consider $(\mathbb Z_3,k)$-sum multisets where the elements of $\mathbb Z_3$ must be produced as the sums of elements. This turns out to be a simple case.

    Similarly to \eqref{apparatus:eq:system} we start by writing down the ways to obtain each of the elements and requiring them to be equal ($\forall \gamma \in \mathbb Z_3 \lambda = \sum (n_\mu (n_{\mu-\gamma}-\delta_{\mu,\mu+\gamma}))$). Adding the $\sum n_\mu = k$ and using $3\lambda = k(k-1)$ we may form a system of equations.
    
    \begin{equation}
        \label{v3:other:eq:system}
        \begin{cases}
            n_0 (n_0-1) + 2 n_1 n_2 = \frac{k(k-1)}{3} \\
            n_1 (n_1-1) + 2 n_2 n_0 = \frac{k(k-1)}{3} \\
            n_2 (n_2-1) + 2 n_0 n_1 = \frac{k(k-1)}{3} \\
            n_0 + n_1 + n_2 = k
        \end{cases}
    \end{equation}

    It can be noticed with ease that \eqref{v3:other:eq:system} possesses symmetry with respect to all the elements of $\mathbb Z_3$. Besides this system can easily be solved explicitly -- valid multisets of $n_\mu$ are $\set{\frac k 3, \frac k 3, \frac k 3}$ and $\set{\frac{k-1}{3}, \frac{k-1}{3}, \frac{k+2}{3}}$.
    
    So, we can conclude that there can be at most one (up to automorphisms) $(\mathbb Z_3, k)$-sum multiset for a given value $k$. Specifically there is one if $3 \mid k$ or $k \equiv 1 \mod 3$ and the multiplicities of elements are $\set{\frac k 3, \frac k 3, \frac k 3}$ and $\set{\frac{k-1}{3}, \frac{k-1}{3}, \frac{k+2}{3}}$ respectively. And there are none if $k \equiv 2 \mod 3$ which eerily reminds of the situations with difference multisets over $\mathbb Z_3$ and $\mathbb Z_2^i$.
    
    Recall remark \ref{dms:remark:abelian}. If we consider any other quasigroup of order 3, it turns out that in every case the difference multisets and sum multisets give raise to either system \eqref{v3:eq:constraints} or the system \eqref{v3:other:eq:system}. There are only 5 quasigroups of order 3 so this can be checked on a case by case basis. We have thus solved the problem for every quasigroup of size 3.
    

	
    \section{Generalizations}
    The concept of difference multisets can be generalized to non-abelian groups or even any other algebraic structures where \emph{differences} can be defined. Such structures are generally known as \emph{quasigroups}---algebras with a unique solution to $\gamma\cdot\mu=\nu$ both when solving for $\gamma$ and when solving for $\mu$. The unique $\gamma$
such that $\gamma\cdot\mu=\nu$ is called the right division $\nu/\mu$. Similarly the left division can be defined.
Without loss of generality we shall consider the right division. 

Let $Q$ be a quasigroup. For a multiset $M$ over $Q$ to be a difference multiset, any element $\gamma\in Q$ must be obtained $\lambda$ times as $\nu/\mu$ where $\nu$, $\mu\in M$. Notice that if $\gamma=\nu/\mu$, then $\nu=\gamma\mu$. The generalization of equation \eqref{apparatus:eq:system} is:

    \begin{equation}
        \label{generalization:eq:system}
        \forall \gamma \in Q \colon \sum (n_\mu(n_{\gamma\mu}-\delta_{\mu,\gamma\mu})) = \lambda
    \end{equation}
which coincides with system \eqref{apparatus:eq:system} in case of loops (quasigroups with an identity element)---if there is an identity element then $\delta_{\gamma0}=\delta{\mu,\gamma\mu}$.

The digression equation \eqref{apparatus:eq:dsystem} generalizes to something more complex:
    \begin{equation}
        \label{generalization:eq:dsystem}
        \forall \gamma \in Q \colon \sum (d_\mu (d_{\gamma\mu}-\frac{v\delta_{\mu,\gamma\mu}}{\sqrt k})-v\delta_{\mu,\gamma\mu}) = -v.
    \end{equation}

However, in the case of loops system \eqref{generalization:eq:dsystem} simplifies to system \eqref{apparatus:eq:dsystem}.

\subsection{Generalized results}

As the equation \eqref{apparatus:eq:dsystem} applies for loops, Theorem \ref{general:theorem:limits} is true for all loops.

Theorem \ref{regular:theorem:regular} is also true for loops as it was proved using equations that hold for loops. However, it also holds if there is only a one-sided identity element.

\begin{theorem}
        \label{generalization:theorem:regular}
        Suppose $Q$ is a quasigroup with right division considered as difference. Then the construction given in Theorem \ref{regular:theorem:regular} produces a difference multiset iff $\exists \gamma \forall \mu \colon \mu=\gamma\mu$.
    \end{theorem}
    
    \begin{proof}
        For $\gamma \in Q$ let $Q_\gamma=\{\mu\in Q\mid \mu=\gamma\mu\}$.
        Denote $v_\gamma = |Q_\gamma|$ and $\overline{v_\gamma} = v - v_\gamma$.
%be the subset of elements for which $\gamma$ behaves like left identity:
%        \begin{equation}
%            \mu=\gamma\mu \iff \mu \in Q_\gamma
%        \end{equation}
        
        We can rewrite equation \eqref{generalization:eq:dsystem} for $\gamma$:
        \begin{equation}
            \label{generalization:eq:dsystem_split}
            \sum\limits_{\mu \in Q_\gamma} (d_\mu (d_\mu-\frac v {\sqrt k})
            + \sum\limits_{\mu \notin Q_\gamma} d_\mu d_{\gamma\mu}
             = v(v_\gamma - 1).
        \end{equation}

        Let us try to construct a difference multiset by selecting element $\nu \in Q$ and setting $d_\nu=\pm(v-1)$ and $d_{\mu\neq\nu}=\mp 1$.
        
        If $\nu \in Q_\gamma$ then $v_\gamma \neq 0$ and equation \eqref{generalization:eq:dsystem_split} becomes
        \begin{equation}
            (v-1)(v-1-\frac v {\sqrt k}) 
             + (v_\gamma - 1)(1 + \frac v {\sqrt k})
             + \overline{v_\gamma}
            = v^2 - v + v \frac v {\sqrt k} (v_\gamma-v)
        \end{equation}
        which is only equal to $v(v_\gamma - 1)$ if $v_\gamma=v$.
        
        If $\nu \notin Q_\gamma$ then $\overline{v_\gamma} \neq 0$ and equation \eqref{generalization:eq:dsystem_split} becomes
        \begin{equation}
            v_\gamma (1 + \frac v {\sqrt k})
             + 2 (1 - v)
             + \overline{v_\gamma} - 2
            = -v + v \frac {v_\gamma} {\sqrt k}
        \end{equation}
        which is only equal to $v(v_\gamma - 1)$ if $v_\gamma=0$.
        
        The above conditions ($v_\gamma$ always being $0$ or $v$) are satisfied only if there is a left identity element $\gamma$.
    \end{proof}

    Similarly it can be shown that left division difference multisets of same structure require a right identity element.

\subsection{Other constructions over quasigroups of size 3}
    \label{sec:v3}
    Considering $\mathbb Z_3$ with difference as the quasigroup operation, we obtain $(\mathbb Z_3,k)$-sum multisets where the elements of $\mathbb Z_3$ must be produced as the sums of elements. This turns out to be a simple case.

    Similarly to \eqref{generalization:eq:system} we start by writing down the ways to obtain each of the elements and requiring them to be equal ($\forall \gamma \in \mathbb Z_3 \colon  \lambda = \sum (n_\mu (n_{\gamma-\mu}-\delta_{\mu,\gamma-\mu}))$). Adding the equation $\sum n_\mu = k$ and using $3\lambda = k(k-1)$ we obtain a system of equations:
    
    \begin{equation}
        \label{generalizations:v3:eq:system}
        \begin{cases}
            n_0 (n_0-1) + 2 n_1 n_2 = \frac{k(k-1)}{3} \\
            n_1 (n_1-1) + 2 n_2 n_0 = \frac{k(k-1)}{3} \\
            n_2 (n_2-1) + 2 n_0 n_1 = \frac{k(k-1)}{3} \\
            n_0 + n_1 + n_2 = k
        \end{cases}.
    \end{equation}

    The system \eqref{generalizations:v3:eq:system} obviously possesses symmetry with respect to all the elements of $\mathbb Z_3$ and this system can easily be solved explicitly---valid multisets of $n_\mu$ are $\set{\frac k 3, \frac k 3, \frac k 3}$ and $\set{\frac{k-1}{3}, \frac{k-1}{3}, \frac{k+2}{3}}$.
    
    We can conclude that there can be at most one (up to automorphisms) $(\mathbb Z_3, k)$-sum multiset for a given value of $k$. Specifically there is one if $3 \mid k$ or $k \equiv 1 \pmod 3$ and the multiplicities of elements are $\set{\frac k 3, \frac k 3, \frac k 3}$ and $\set{\frac{k-1}{3}, \frac{k-1}{3}, \frac{k+2}{3}}$ respectively. And there are none if $k \equiv 2 \pmod 3$ similarly to the situation with difference multisets over $\mathbb Z_3$.
    
    If we consider any other quasigroup of order 3, it turns out that in every case the difference multisets and sum multisets give raise to either system \eqref{v3:eq:constraints} or the system \eqref{generalizations:v3:eq:system}. There are only 5 quasigroups of order 3 so this can be checked on a case by case basis. We have thus solved the problem for every quasigroup of size 3.
    


    \section{Summary and conclusions}
	We have found what are the difference multisets if the parameter values are small. Here we present a list of our findings. Some trivial cases that formally satisfy the constraints (e.g. some produce every element 0 times) and quasigroup cases are also included as those have helped spotting patterns.

\begin{tabular}{llr}
\toprule
    Parameters & Difference multisets \\
\midrule
    $k = 0$ & Empty multiset. \\
    $k = 1$ & Take single element, works for $v \geq 1$. \\
    $v = 0$ & Empty multiset. \\
    $v = 1$ & Take the identity $k$ times for any $k$. \\
    $v = 2$ & Covered in Section \ref{sec:z2i} as a case of $\mathbb Z_2^i$. \\
    $v = 3$ & See section \ref{sec:v3}. \\
    $G=\mathbb Z_2^i$ & See section \ref{sec:z2i}, possibly incomplete. \\
    $G=\mathbb Z_3$ & See section \ref{sec:z3}. \\
\bottomrule
\end{tabular}


The case of difference multisets over $\mathbb Z_3$ 
(theorem \ref{v3:theorem:loeschian}) shows that not only the very 
trivial cases can be solved explicitly. Although it is not 
straightforward to generalize our methods for arbitrary $\mathbb Z_i$, 
solving the problem for odd prime values of $i$ seems in the realm of 
possibility.

Theorem \ref{general:theorem:limits} and proposition \ref{general:theorem:even_cyclic}
narrows the space of options that has to be considered in computer 
searches thus allowing to inspect a wider range of difference multisets
and draw conclusions through observations.

The mathematical apparatus we used is also applicable for many other 
cases. The results presented in this paper are the ones that are in some 
sense complete or general. Other than these cases the system 
\eqref{apparatus:eq:dsystem} (or alternative forms) can be utilised to 
investigate some difference multisets or sets of their digressions in many 
small quasigroups.


	\appendix
	\section{Calculation details for section \ref{sec:z2i}}
	\label{sec:appendix_z2_i}
	In this appendix we show how the sums in equations \eqref{z2i:eq:appendixable1} and 
\eqref{z2i:eq:appendixable2} are calculated.

\subsection{Auxiliary results}

For ease of reading let us repeat from \ref{sec:z2i} that 
$i\in \set{1,2,\ldots,m}$ and the definition of $\xi_i$ is as follows:

\begin{equation}
	\xi_i = \frac{1-(-2)^{i+1}} 3.
\end{equation}

Evaluate the following expressions:

\begin{equation}
	\label{appendix:eq:xi1}
	\xi_i^2
	 = \frac 1 9 \left(
		1 + (-2)^{i+2} + 2^{2i+2}
	 \right);
\end{equation}

\begin{equation}
	\label{appendix:eq:xi2}
	\xi_i (\xi_i - 2 \xi_s)
	 = -\frac 1 9 \left(
		1 + (-2)^{s+2} - 2^{2i+2} - (-2)^{s+i+3}
	 \right);
\end{equation}

\begin{equation}
	\label{appendix:eq:xi3}
	\xi_{r+1} (\xi_{r+1} + 2(-2)^r)
	= \frac 1 9 \left(
		1 + (-2)^{r+1} + (-2)^{2r+3}
	\right)
\end{equation}

We will also need the values of the following sums:

\begin{equation}
	\label{appendix:eq:sum1}
	\sum\limits_{i<s} 2^{-i} = 1 - 2^{1-s};
\end{equation}

\begin{equation}
	\label{appendix:eq:sum2}
	\sum\limits_{i<s}  2^{-i} (-2)^{s+2}
	= (-1)^s (2^{s+2} - 2^3);
\end{equation}

\begin{equation}
	\label{appendix:eq:sum3}
	\sum\limits_{i<s} 2^{-i} 2^{2i+2}
	= \sum\limits_{i<s} 2^{i+2}
	= 2^{s+2} - 2^3;
\end{equation}

\begin{equation}
	\label{appendix:eq:sum4}
	\sum\limits_{i<s} 2^{-i} (-2)^{s+i+3}
	= 2^{s+3} (-1)^{s+1} \sum\limits_{i<s} (-1)^i
	= 2^{s+2} (1+(-1)^s);
\end{equation}

\begin{equation}
	\label{appendix:eq:sum5}
	\sum\limits_{i \leq r} 2^{-i} = 1 - 2^{-r};
\end{equation}

\begin{equation}
	\label{appendix:eq:sum6}
	\sum\limits_{i \leq r} 2^{-i} (-2)^{i+2}
	= -2 (1 - (-1)^r);
\end{equation}

\begin{equation}
	\label{appendix:eq:sum7}
	\sum\limits_{i \leq r} 2^{-i} 2^{2i+2}
	= 2^3 (2^r - 1).
\end{equation}

\subsection{Evaluation of expression in \eqref{z2i:eq:appendixable1}}

Let us start by separating case of $i=s$ out of the sum, factoring out
$2^m$ and grouping the terms. We can evaluate the obtained
expression using the notes on $\xi_i$ from the previous section:

\begin{equation}
	\begin{split}
		& \sum\limits_{i<s} 2^{m-i} \xi_i^2 - 2 \xi_s \sum\limits_{i \leq s}  2^{m-i} \xi_i \\
		= & 2^m \left[
			-2^{1-s} \xi_s^2 
			+ \sum\limits_{i<s} 2^{-i} \xi_i (\xi_i - 2\xi_s)
		\right] \\
		= & - \frac{2^m}9 \left[
			2^{1-s} (1-(-2)^{s+1})^2
			+ \sum\limits_{i<s} 2^{-i} \left(
				1 + (-2)^{s+2} - 2^{2i+2} - (-2)^{s+i+3}
			\right)
		\right].
	\end{split}
\end{equation}

The sum can be expanded in terms from equations 
\eqref{appendix:eq:sum1}--\eqref{appendix:eq:sum4} and evaluated
as follows:

\begin{equation}
	\begin{split}
		& \sum\limits_{i<s} 2^{-i} \left(
			1 + (-2)^{s+2} - 2^{2i+2} - (-2)^{s+i+3}
		\right) \\
		= & -2^{s+3}  - 2^{1-s} - 2^3(-1)^s + 9.
	\end{split}
\end{equation}

We can now finish the calculation:

\begin{equation}
	\begin{split}
		& \sum\limits_{i<s} 2^{m-i} \xi_i^2 - 2 \xi_s \sum\limits_{i \leq s}  2^{m-i} \xi_i \\
		= & - \frac{2^m}9 \left[
			2^{1-s} (1+(-2)^{s+2} + 2^{2s+2})
			- 2^{s+3}  - 2^{1-s} - 2^3(-1)^s + 9
		\right] \\
		= & -2^m.
	\end{split}
\end{equation}

\subsection{Evaluation of expression in \eqref{z2i:eq:appendixable2}}

Start by rearranging the terms and factoring out the $2^m$ and $2^{-r}$.

\begin{equation}
	\label{appendix:eq:2step1}
	\begin{split}
		& \sum\limits_{i\leq r} 2^{m-i} \xi_i^2
			+ (2^{m-r}-2) \xi_{r+1}^2 + 2\xi_{r+1}((-1)^r 2^m + \xi_{r+1}) \\
			= & \sum\limits_{i\leq r} 2^{m-i} \xi_i^2
				+ 2^{m-r}\xi_{r+1}^2 + 2^{m+1} (-1)^r \xi_{r+1} \\
			= & 2^m \left[
				\sum\limits_{i\leq r} 2^{-i} \xi_i^2
					+ 2^{-r}\xi_{r+1} \left(
						\xi_{r+1} + 2 (-2)^r 
					\right)
			\right].
	\end{split}
\end{equation}

The sum can be evaluated using results \eqref{appendix:eq:xi1} and
\eqref{appendix:eq:sum5}--\eqref{appendix:eq:sum7}:

\begin{equation}
	\begin{split}
		& \sum\limits_{i\leq r} 2^{-i} \xi_i^2 \\
		= & \frac 1 9 \sum\limits_{i\leq r} 2^{-i} \left(
			1 + (-2)^{i+2} + 2^{2i+2}
		\right) \\
		= & \frac 1 9 \left(
			2^{r+3} - 2^{-r} + 2(-1)^r - 9
		\right).
	\end{split}
\end{equation}

The other term was precalculated in \eqref{appendix:eq:xi3}. And we obtain 
the result:

\begin{equation}
	\begin{split}
		& \sum\limits_{i\leq r} 2^{m-i} \xi_i^2
			+ (2^{m-r}-2) \xi_{r+1}^2 + 2\xi_{r+1}((-1)^r 2^m + \xi_{r+1}) \\
		= & \frac {2^m} 9 \left[
			2^{r+3} - 2^{-r} + 2(-1)^r - 9
			+ 2^{-r} \left(
				1 + (-2)^{r+1} + (-2)^{2r+3}
			\right)
		\right] \\
		= & -2^m.
	\end{split}
\end{equation}

	
	\section*{Acknowledgements}
	We would like to express our gratitude to Anna Jansone for coming
up with the concise proof of $\mathbb Z_2^2$ case in theorem 
\ref{z2i:theorem:exclusive} \cite{jansone_z2_2} and Mathematics 
Stack Exchange user B. Mehta who helped to complete the proof of
lemma \ref{v3:lemma:square} \cite{math_se_mehta}.
    
	\bibliographystyle{amsplain}
	\bibliography{difference_multisets}
\end{document}
